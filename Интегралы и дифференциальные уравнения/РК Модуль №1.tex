\input{../../preamble2.tex}

\begin{document}

\begin{titlepage}
    \vspace*{0pt}
    \vfill
    \centering
    \Huge\textbf{Интегралы и дифференциальные уравнения} \\[7pt]
    \Large\textbf{Рубежный контроль} \\
    \large 2 семестр | Модуль №1 \\ 
    \vfill
    \begin{flushright}
        \normalsize GitHub: \href{https://github.com/malyinik}{malyinik} \\
    \end{flushright}
    \normalsize 2024 г.
\end{titlepage}
\newpage

\tableofcontents
\newpage

\section{Вопросы, оцениваемые в 1 балл}

\subsection{Сформулировать определение первообразной}

\begin{definition}\hlabel{опр: первообразная}
    Функция $F(x)$ называется \textbf{первообразной} функции $f(x)$ на интервале $(a;b)$, если $F(x)$ дифференцируема на $(a;b)$ и $\forall x \in (a;b)\colon$
    \begin{gather}
        \boxed{F'(x) = f(x)}
    \end{gather}
\end{definition}

\subsection{Сформулировать определение неопределённого интеграла}

\begin{definition}\hlabel{опр: неопределённый интеграл}
    Множество первообразных функции $f(x)$ на $(a;b)$ называется\break\textbf{неопределённым интегралом}.
    \begin{gather}
        \boxed{\int f(x)\, dx = F(x) + C}
    \end{gather}
    $\int$ --- знак интеграла\\
    $f(x)$ --- подынтегральная функция\\
    $f(x)\, dx$ --- подынтегральное выражение\\
    $x$ --- переменная\\
    $F(x) + C$ --- множество первообразных\\
    $C$ --- произвольная константа
\end{definition}

\subsection{Сформулировать определение определённого интеграла}
Пусть функция $y = f(x)$ определена на $[a;b]$.\\
Рассмотрим произвольное разбиение $[a;b]$. В каждом из отрезков разбиения $[x_{i-1}; x_i]$ выберем точку $\xi_i,\ i = \overline{1, n}$. Составим сумму
\begin{gather}\hlabel{ф: интегральная сумма}
    \boxed{S_n = \sum_{i=1}^{n}f(\xi_i)\cdot \Delta x_i}
\end{gather}

\begin{definition}\hlabel{опр: определённый интеграл}
    \textbf{Определённым интегралом} от функции $y=f(x)$ на $[a;b]$ называется \underline{конечный} предел интегральной суммы (\ref{ф: интегральная сумма}), когда число отрезков разбиения растёт, а их длины стремятся к нулю.
    \begin{gather}\hlabel{ф: определённый интеграл}
        \boxed{\int_{a}^{b} f(x)\, dx = \lim_{\lambda \to 0} \sum_{i=1}^{n} f(\xi_i)\cdot \Delta x_i}
    \end{gather}
    Предел (\ref{ф: определённый интеграл}) не зависит от способа разбиения отрезка $[a;b]$ и выбора точек $\xi_i,\ \overline{1, n}$.\\
    $f(x)$ --- подынтегральная функция\\
    $f(x)\, dx$ --- подынтегральное выражение\\
    $\displaystyle\int_{a}^{b}$ --- знак определённого интеграла\\
    $a$ --- нижний предел интегрирования\\
    $b$ --- верхний предел интегрирования
\end{definition}

\subsection{Сформулировать определение интеграла с переменным верхним пределом}

Пусть $f(x)$ непрерывна на $[a;b]$. Рассмотрим $\int_{a}^{b} f(x)\, dx$. Закрепим нижний предел интегрирования $a$. Изменяем верхний предел интегрирования $b$, чтобы подчеркнуть изменение верхнего предела интегрирования. \\
$\displaystyle b \longrightarrow x\quad x \in [a;b]\quad [a;x] \subset [a;b]\quad I(x) = \int_{a}^{x} f(t)\, dt$. \\
\begin{definition}
    \textbf{Определённым интегралом с переменным верхним пределом интегрирования} от непрерывной функции $f(x)$ на $[a;b]$ называется интеграл вида
    \begin{gather*}
        \boxed{I(x) = \int_{a}^{x} f(t)\, dt},\ \text{где } x\in [a;b]
    \end{gather*}
\end{definition}

\begin{figure}[h]
    \centering
    \begin{tikzpicture}
        \tkzInit[xmin=-0.5, xmax=6, ymin=-0.5, ymax=2]
        \tkzDrawX[label={t}] \tkzDrawY
        \node[below left] at (0, 0) {$0$};
        \filldraw[pattern=north east lines, pattern color=gray] (1, 0) -- (1, 1) .. controls (1.9, 2.1) and (3, 1.3) .. (3, 1.45) -- (3, 0) -- cycle;
        \draw[very thick] (1, 1) .. controls (2, 2.5) and (4, 0.5) .. (5, 1.5) node[right]{$y=f(t)$};
        \draw[thick] (1, 0) node[below]{$a$} -- (1, 1);
        \draw[thick, dashed] (3, 0) node[below]{$x$} -- (3, 1.45);
        \draw[thick, dashed] (5, 0) node[below]{$b$} -- (5, 1.5);
    \end{tikzpicture}
\end{figure}
$I(x)$ --- переменная площадь криволинейной трапеции с основанием $[a;x] \subset [a;b]$.

\subsection{Сформулировать определение несобственного интеграла 1-го рода}
Пусть $y = f(x)$ определена на $[a; +\infty)$, интегрируема на $[a;b]\subset [a; +\infty)$. Тогда определена функция
\begin{gather}
    \boxed{\Phi (b) = \int_{a}^{b} f(x)\dd{x}} \text{ на } [a; +\infty)
\end{gather}
как определённый интеграл с переменным верхним пределом интегрирования.\\
\begin{definition}\hlabel{опр: несобственный интеграл 1 рода}
    Предел функции $\Phi(b)$ при $b\to +\infty$ называется несобственным интегралом от функции $f(x)$ по бесконечному промежутку $[a; +\infty)$ или \textbf{несобственным интегралом 1-го рода} и обозначается
    \begin{gather}
        \boxed{\int_{a}^{+\infty} f(x)\dd{x} = \lim_{b \to +\infty} \Phi(b) = \lim\limits_{b \to +\infty} \int_{a}^{b}f(x)\dd{x}} 
    \end{gather}
\end{definition}

\newpage
\subsection{Сформулировать определение несобственного интеграла 2-го рода}
Пусть функция $f(x)$ определена на полуинтервале $[a;b)$, а в точке $x = b$ терпит разрыв 2-го рода. Предположим, что функция $f(x)$ интегрируема на $[a; \eta] \subset [a;b)$. Тогда на $[a; b)$ определена функция
\begin{gather}
    \Phi (\eta) = \int_{a}^{\eta} f(x)\dd{x}
\end{gather}
как интеграл с переменным верхним пределом. \\
\begin{definition}
    Предел функции $\Phi (\eta)$ при $\eta \to b-$ называется несобственным интегралом от неограниченной функции $f(x)$ на $[a;b)$ или \textbf{несобственным интегралом 2-го рода} и обозначается
    \begin{gather}
        \boxed{\int_{a}^{b} f(x)\dd{x} = \lim\limits_{\eta \to b-} \Phi (\eta) = \lim\limits_{\eta \to b-} \int_{a}^{\eta} f(x)\dd{x}}
    \end{gather}
\end{definition}

\subsection{Сформулировать определение сходящегося несобственного интеграла 1-го рода}

\begin{definition}
    \[
        \int_{a}^{+\infty} f(x)\dd{x} = \lim_{b \to +\infty} \Phi(b) = \lim\limits_{b \to +\infty} \int_{a}^{b}f(x)\dd{x}
    \]
    Если предел в правой части равенства существует и конечен, то несобственный интеграл в левой части равенства \textbf{сходится}.
\end{definition}

\subsection{Сформулировать определение абсолютно сходящегося несобственного интеграла 1-го рода}

\begin{definition}
    Если наряду с несобственным интегралом от функции $f(x)$ по бесконечному промежутку $[a;+\infty)$ сходится и несобственный интеграл от функции $|f(x)|$ по этому же промежутку, то первый несобственный интеграл называется \textbf{сходящимся абсолютно}.
    \begin{gather*}
        \boxed{\begin{aligned}
            &\text{несобственный интеграл} \\
            &\text{от $f(x)$ сходится абсолютно} 
        \end{aligned}} = \boxed{\begin{aligned}
            &\text{несобственный интеграл}  \\
            &\text{от $f(x)$ сходится}
        \end{aligned}} + \boxed{\begin{aligned}
            &\text{несобственный интеграл} \\
            &\text{от $|f(x)|$ сходится}
        \end{aligned}}
    \end{gather*}
\end{definition}

\newpage
\subsection{Сформулировать определение условно сходящегося несобственного интеграла 1-го рода}

\begin{definition}
    Если несобственный интеграл от функции $f(x)$ по бесконечному промежутку $[a;+\infty)$ сходится, а несобственный интеграл от функции $|f(x)|$ по этому же промежутку расходится, то первый несобственный интеграл называется \textbf{сходящимся условно}.
    \begin{gather*}
        \boxed{\begin{aligned}
                &\text{несобственный интеграл} \\
                &\text{от $f(x)$ сходится условно} 
            \end{aligned}} = \boxed{\begin{aligned}
                &\text{несобственный интеграл}  \\
                &\text{от $f(x)$ сходится}
            \end{aligned}} + \boxed{\begin{aligned}
                &\text{несобственный интеграл} \\
                &\text{от $|f(x)|$ расходится}
            \end{aligned}}
    \end{gather*}
\end{definition}

\subsection{Сформулировать определение сходящегося несобственного интеграла 2-го рода}

\begin{definition}
    \[
        \int_{a}^{b} f(x)\dd{x} = \lim\limits_{\eta \to b-} \Phi (\eta) = \lim\limits_{\eta \to b-} \int_{a}^{\eta} f(x)\dd{x}
    \]
    Если предел в правой части равенства существует и конечен, то несобственный интеграл от неограниченной функции $f(x)$ по $[a;b)$ \textbf{сходится}.
\end{definition}

\subsection{Сформулировать определение абсолютно сходящегося несобственного интеграла 2-го рода}

\begin{definition}
    Если несобственный интеграл от неограниченной функции $f(x)$ при $x \to b-$ по промежутку $[a;b)$ сходится и несобственный интеграл функции $|f(x)|$ по этому же промежутку сходится, то первый из несобственных интегралов \textbf{сходится абсолютно}.
\end{definition}

\subsection{Сформулировать определение условно сходящегося несобственного интеграла 2-го рода}

\begin{definition}
    Если несобственный интеграл от неограниченной функции $f(x)$ при $x\to b-$ по промежутку $[a;b)$, сходится, а несобственный интеграл от функции $|f(x)|$ по этому же промежутку расходится, то первый из несобственных интегралов \textbf{сходится условно}.
\end{definition}

\newpage
\section{Вопросы, оцениваемые в 3 балла}

\subsection{Сформулировать и доказать теорему об оценке определённого интеграла}

\begin{theorem}[Об оценке определённого интеграла]
    Пусть функции $f(x)$ и $g(x)$ интегрируемы на $[a;b]$ и $\forall x \in [a;b]\colon m \leqslant f(x) \leqslant M,\ {g(x) \geqslant 0},\break m,\, M \in \R$. Тогда
    \begin{gather*}
        \boxed{m \int_{a}^{b} g(x)\, dx \leqslant \int_{a}^{b} f(x)\, g(x)\, dx \leqslant M \int_{a}^{b} g(x)\, dx}
    \end{gather*}
\end{theorem}
\begin{proof}
    Так как $\forall x \in [a;b]$ верны неравенства
    \begin{gather*}
        \begin{aligned}
            m \leqslant f(x) &\leqslant M\quad | \cdot g(x) \\
            g(x) &\geqslant 0\qquad m, M \in \R
        \end{aligned} \\[1ex]
        m\cdot g(x) \leqslant f(x)\cdot g(x) \leqslant M \cdot g(x)
    \end{gather*}
    По теореме \ref{т: интегрирование неравенства} и \ref{т: линейная комбинация интегрируемых}:
    \begin{gather*}
        m \int_{a}^{b} g(x) \leqslant \int_{a}^{b} f(x)\, g(x)\, dx \leqslant M \int_{a}^{b} g(x)\, dx
    \end{gather*}
\end{proof}

\subsection{Сформулировать и доказать теорему о среднем}

\begin{theorem}[О среднем значении для определённого интеграла]\hlabel{т: среднее значение опр. интеграла}
    Если $f(x)$ непрерывна на $[a;b]$, то
    \begin{gather*}
        \exists\, c \in [a;b]\colon f(c) = \frac{1}{b-a} \int_{a}^{b} f(x)\, dx
    \end{gather*}
\end{theorem}
\begin{proof}
    Так как функция $y=f(x)$ непрерывна на $[a;b]$, то по теореме \textit{Вейерштрасса} она достигает своего наибольшего и наименьшего значения. \\
    То есть $\exists\, m, M \in \R,\ \forall x \in [a;b]\colon m \leqslant f(x) \leqslant M$ \\
    По теореме \ref{т: интегрирование неравенства}:
    \begin{gather*}
        \int_{a}^{b} m\, dx \leqslant \int_{a}^{b} f(x)\, dx \leqslant \int_{a}^{b} M\, dx
    \end{gather*}
    По теореме \ref{т: линейная комбинация интегрируемых}:
    \begin{gather*}
        m \int_{a}^{b} dx \leqslant \int_{a}^{b} f(x)\, dx \leqslant M \int_{a}^{b} dx
    \end{gather*}
    По теореме \ref{т: опр. интеграл и константа}:
    \begin{gather*}
        m(b-a) \leqslant \int_{a}^{b} f(x)\, dx \leqslant M(b-a)\quad | : (b-a)
    \end{gather*}
    Так как функция $y=f(x)$ непрерывна на $[a;b]$, то по теореме \textit{Больцано-Коши} она принимает все свои значения между наибольшим и наименьшим значением.
    \begin{gather*}
        m \leqslant \frac{1}{b-a} \int_{a}^{b} f(x)\, dx \leqslant M
    \end{gather*}
    По теореме \textit{Больцано-Коши} $\exists\, c \in [a;b]\colon$
    \begin{gather*}
        f(c) = \frac{1}{b-a} \int_{a}^{b} f(x)\, dx
    \end{gather*}
\end{proof}

\subsection{Сформулировать и доказать теорему о производной интеграла с переменным верхним пределом}

\begin{theorem}[О производной $I(x)$]\hlabel{т: производная I(x)}
    Если функция $y=f(x)$ непрерывна на $[a;b]$, то $\forall x \in [a;b]$ верно равенство
    \begin{gather*}
        \boxed{\big(I(x)\big)' = \left(\int_{a}^{x} f(t)\, dt\right)' = f(x)}
    \end{gather*}
\end{theorem}
\begin{proof}
    \begin{gather*}
        \big(I(x)\big)' = \lim\limits_{\Delta x \to 0} \frac{\Delta I(x)}{\Delta x} \xlongequal{\text{Т\ref{т: непрерывность I(x)}}} \lim\limits_{\Delta x \to 0} \frac{f(c)\cdot \Delta x}{\Delta x} = \lim\limits_{\Delta x \to 0} f(c) \xlongequal{\text{$*$}} f(x)
    \end{gather*}
    \begin{tikzpicture}[thick, >=latex]
        \node[] at (-2.5, 0) {$*\colon$};
        \draw (-2, 0.1) -- (-2, -0.1) node[below]{$a$};
        \draw (2, 0.1) -- (2, -0.1) node[below]{$b$};
        \draw (-2, 0) -- (2, 0);
        \draw[->] (0.65, 0) node[below]{\scriptsize${x + \Delta x}$} .. controls (0.40, 0.5) and (-0.3, 0.5) .. (-0.65, 0) node[below]{\scriptsize$x$};
        \draw (0, 0.1) -- (0, -0.1);
        \node[below] at (0, 0) {\scriptsize$c$};
        \node[] at (6, 0) {при $\Delta x \to 0\quad x + \Delta x \to x\quad c \to x$};
    \end{tikzpicture} 
\end{proof}
\begin{corollary}
    Функция $I(x)$ --- первообразная функции $f(x)$ на $[a;b]$, так как по теореме \ref{т: производная I(x)} $\big(I(x)\big)' = f(x)$.
\end{corollary}

\newpage
\subsection{Сформулировать и доказать теорему Ньютона - Лейбница}

\begin{theorem}
    Пусть функция $f(x)$ --- непрерывна на $[a;b]$. Тогда
    \begin{gather*}
        \boxed{\int_{a}^{b} f(x)\, dx = \eval{F(x)}_{a}^{b} =  F(b) - F(a)}
    \end{gather*}
    где $F(x)$ --- первообразная $f(x)$.
\end{theorem}
\begin{proof}
    Пусть $F(x)$ первообразная $f(x)$ на $[a;b]$. По следствию из теоремы \ref{т: производная I(x)} $I(x)$ --- первообразная $f(x)$ на $[a;b]$. \\
    По свойству первообразной: 
    \begin{gather*}
        \begin{aligned}
            & I(x) - F(x) = C \\ 
            & I(x) = F(x) + C,\ \text{где } C \text{ --- } const
        \end{aligned} \\[1ex]
        \int_{a}^{x} f(t)\, dt = F(x) + C,\ \text{где } C \text{ --- } const \tag{$\vee$}
    \end{gather*}
    $\bullet\ x = a\colon$
    \begin{align*}
        \int_{a}^{a} f(t)\, dt &= F(a) + C \\
        0 &= F(a) + C \\
        C &= -F(a)
    \end{align*}
    $C = -F(a)$ подставим в $(\vee)\colon$
    \begin{gather*}
        \int_{a}^{x} f(t)\, dt = F(x) - F(a)
    \end{gather*}
    $\bullet\ x = b\colon$
    \begin{gather*}
        \boxed{\int_{a}^{b} f(t)\, dt = F(b) - F(a)}
    \end{gather*}
\end{proof}

\newpage
\subsection{Сформулировать и доказать теорему об интегрировании по частям в определённом интеграле}

\begin{theorem}
    Пусть функции $u = u(x)$ и $v = v(x)$ непрерывно дифференцируемы на $[a;b]$. Тогда имеет место равенство
    \[
        \boxed{\int_{a}^{b} u\, dv = \eval{uv}_{a}^{b} - \int_{a}^{b} v\, du}
    \]
\end{theorem}
\begin{proof}
    Рассмотрим произведение функций $u\cdot v$. \\
    Дифференцируем:
    \begin{align*}
        d(u\cdot v) &=  v\, du + u\, dv \\ 
        u\, dv &= d(u v) - v\, du
    \end{align*}
    Интегрируем:
    \begin{align*}
        \int_{a}^{b} u\, dv = \int_{a}^{b} \big(d(uv) - v\, du\big) = \int_{a}^{b} d(uv) - \int_{a}^{b} v\, du = \eval{uv}_{a}^{b} - \int_{a}^{b} v\, du
    \end{align*}
\end{proof}

\newpage
\subsection{Сформулировать и доказать признак сходимости по неравенству для несобственных интегралов 1-го рода}

\begin{theorem}[Признак сходимости по неравенству]\hlabel{т: сходимость по неравенству}
    Пусть функции $f(x)$ и $g(x)$ интегрируемы на $[a;b]\subset [a; +\infty)$, причём \[ \forall x \geqslant a\colon 0 \leqslant f(x) \leqslant g(x) \] Тогда:
    \begin{enumerate}
        \item Если $\int_{a}^{+\infty} g(x)\dd{x}$ сходится, то $\int_{a}^{+\infty} f(x)\dd{x}$ --- сходится
        \item Если $\int_{a}^{+\infty} f(x)\dd{x}$ расходится, то $\int_{a}^{+\infty} g(x)\dd{x}$ --- расходится
    \end{enumerate}
    %рисунок
\end{theorem}
\begin{proof}
    $\int_{a}^{+\infty} g(x)\dd{x}$ --- сходится $\Rightarrow$ по \stackon{определению}{\footnotesize{(\textbf{опр. \ref{опр: несобственный интеграл 1 рода}})}} несобственного интеграла 1-го рода
    \[
        \int_{a}^{+\infty} g(x)\dd{x} = \lim\limits_{b \to +\infty} \int_{a}^{b} g(x)\dd{x} = C\quad C \text{ --- число}
    \]
    Так как $\forall x \geqslant a\colon g(x) \geqslant 0$
    \[
        \Phi (b) = \int_{a}^{b} g(x)\dd{x} \leqslant C,\quad b > a
    \]
    По условию: $\forall x \geqslant a\colon 0 \leqslant f(x) \leqslant g(x)$\\
    Интегрируем:
    \[
        \int_{a}^{b} f(x)\dd{x} \leqslant \int_{a}^{b} g(x)\dd{x} \leqslant C
    \]
    Так как $f(x) \geqslant 0,\ \forall x \geqslant a$ и $b > a$, то функция
    \[
        \Psi (b) = \int_{a}^{b} f(x)\dd{x} \text{ монотонно возрастает и ограничена сверху}
    \]
    \textbf{Утверждение:} монотонная и ограниченная сверху функция при $x \to +\infty$ имеет конечный предел.\\
    По утверждению функция $\Psi(b)$ имеет конечный предел при $x\to +\infty$, то есть
    \[
        \int_{a}^{+\infty} f(x)\dd{x} = \lim\limits_{b \to +\infty}  \Psi(b) = \lim\limits_{b \to +\infty} \int_{a}^{b}f(x)\dd{x} \text{ --- конечный предел}
    \]
\end{proof}
\begin{proof}[Метод от противного]
    \textbf{Дано:} $\int_{a}^{+\infty}f(x)\dd{x}$ --- расходится\\
    Предположим, что $\int_{a}^{+\infty}g(x)\dd{x}$ --- сходится\\
    Тогда по первой части теоремы: \vspace{-0.5\topsep}
    \[
        \int_{a}^{+\infty} f(x)\dd{x} \text{ --- сходится} 
    \]
    А это противоречит условию теоремы $\Rightarrow \int_{a}^{+\infty} g(x)\dd{x}$ --- расходится
\end{proof}

\subsection{Сформулировать и доказать предельный признак сравнения для несобственных интегралов 1-го рода}

\begin{theorem}[Предельный признак сходимости]
    Пусть $f(x)$ и $g(x)$ интегрируемы на $[a;b] \subset [a; +\infty)$ и $\forall x \geqslant a\colon f(x) \geqslant 0,\ g(x) > 0$. Если существует конечный предел:
    \begin{gather}
        \lim\limits_{x \to +\infty} \frac{f(x)}{g(x)} = \lambda > 0
    \end{gather}
    то $\displaystyle\int_{a}^{+\infty} f(x)\dd{x}$ и $\displaystyle\int_{a}^{+\infty} g(x)\dd{x}$ сходятся или расходятся одновременно.
\end{theorem}
\begin{proof}
    Из (5) $\Rightarrow$ по определению предела:
    \[ \forall \varepsilon > 0,\ \exists\, M(\varepsilon) > 0\colon \forall x > M\ \Rightarrow \left| \frac{f(x)}{g(x)} - \lambda \right| < \varepsilon \] \vspace{-\topsep}
    \begin{gather*}
        -\varepsilon < \frac{f(x)}{g(x)} - \lambda < \varepsilon\\
        \lambda - \varepsilon < \frac{f(x)}{g(x)} < \lambda + \varepsilon\\
        (\lambda - \varepsilon)\cdot g(x) < f(x) < (\lambda + \varepsilon)\cdot g(x)\quad \forall x > M \tag{$*$}
    \end{gather*}
    \fbox{1 шаг} Рассмотрим $f(x) < (\lambda + \varepsilon) g(x)$\\
    Интегрируем:
    \[ 
        \int_{a}^{+\infty} f(x)\dd{x} < (\lambda + \varepsilon) \int_{a}^{+\infty} g(x)\dd{x}
    \]
    Число $(\lambda + \varepsilon)$ не влияет на сходимость/расходимость несобственного интеграла.\\[1ex]
    Пусть $\int_{a}^{+\infty} g(x)\dd{x}$ --- сходится, тогда:
    \[
        (\lambda + \varepsilon) \int_{a}^{+\infty}g(x)\dd{x} \text{ --- сходится}
    \]
    По теореме \ref{т: сходимость по неравенству} $\int_{a}^{+\infty} f(x)\dd{x}$ --- сходится.\\[1ex]
    Пусть $\int_{a}^{+\infty} f(x)\dd{x}$ расходится, тогда по теореме \ref{т: сходимость по неравенству}
    \[
        (\lambda + \varepsilon) \int_{a}^{+\infty}g(x)\dd{x} \text{ --- расходится}\ \Rightarrow\ \int_{a}^{+\infty} g(x)\dd{x} \text{ --- расходится}
    \]
    \fbox{2 шаг} Рассмотрим $(\lambda - \varepsilon)\cdot g(x) < f(x)$\\
    Интегрируем:
    \[
        (\lambda - \varepsilon) \int_{a}^{+\infty} g(x)\dd{x} < \int_{a}^{+\infty} f(x)\dd{x}
    \]
    $(\lambda - \varepsilon)$ не влияет на сходимость/расходимость несобственного интеграла\\[1ex]
    Пусть $\int_{a}^{+\infty} f(x)\dd{x}$ --- сходится, тогда по теореме \ref{т: сходимость по неравенству} \[
        (\lambda - \varepsilon)\int_{a}^{+\infty} g(x)\dd{x} \text{ --- сходится}\ \Rightarrow\ \int_{a}^{+\infty} g(x)\dd{x} \text{ --- сходится}
    \]
    Пусть $(\lambda - \varepsilon) \int_{a}^{+\infty} g(x)\dd{x}$ --- расходится, тогда $\int_{a}^{+\infty} g(x)\dd{x}$ --- расходится\\[1ex]
    По теореме \ref{т: сходимость по неравенству} $\int_{a}^{+\infty} f(x)\dd{x}$ расходится, тогда
    \[
        \int_{a}^{+\infty} f(x)\dd{x} \text{ и } \int_{a}^{+\infty} g(x)\dd{x} \text{ сходятся и расходятся одновременно}
    \]
\end{proof}

\subsection{Сформулировать и доказать признак абсолютной сходимости для несобственных интегралов 1-го рода}

\begin{theorem}[Признак абсолютной сходимости]
    Пусть функция $f(x)$ знакопеременна на $[a;+\infty)$. Если функции $f(x)$ и $|f(x)|$ интегрируемы на любом отрезке $[a;b] \subset [a; +\infty)$ и несобственный интеграл от функции $|f(x)|$ по бесконечному промежутку $[a; +\infty)$ сходится, то сходится и несобственный интеграл от функции $f(x)$ по $[a; +\infty)$, причём абсолютно.
\end{theorem}
\begin{proof}
    Так как $\forall x \in [a; +\infty)$ верно неравенство
\begin{align*}
    -|f(x)| \leqslant f(x) &\leqslant |f(x)|\quad \Big| + |f(x)| \\
    0 \leqslant f(x) + |f(x)| &\leqslant 2|f(x)|
\end{align*}
По условию $\int_{a}^{+\infty} |f(x)| \dd{x}$ сходится $\Rightarrow\ 2\int_{a}^{+\infty} |f(x)|\dd{x}$ сходится.\\[1ex]
По теореме \ref{т: сходимость по неравенству} (\textit{признак сходимости по неравенству}):
\[
    \int_{a}^{+\infty} \big(f(x) + |f(x)|\big)\dd{x} \text{ --- сходится}
\]
Рассмотрим
\[
    \int_{a}^{+\infty} f(x)\dd{x} = \underbrace{\int_{a}^{+\infty} \big(f(x) + |f(x)|\big)\dd{x}}_{\text{сх-ся по Т\ref{т: сходимость по неравенству}}} - \underbrace{\int_{a}^{+\infty} |f(x)|\dd{x}}_{\text{сх-ся по условию}}
\]
По определению сходящегося несобственного интеграла $\Rightarrow\ \int_{a}^{+\infty} f(x)\dd{x}$ сходится\\
По определению абсолютной сходимости $\Rightarrow\ \int_{a}^{+\infty} f(x)\dd{x}$ сходится абсолютно
\end{proof}

\newpage
\subsection{Вывести формулу для вычисления площади криволинейного сектора, ограниченного лучами $\varphi = \alpha,\ \varphi = \beta$ и кривой $\rho = \rho(\varphi)$}

\begin{enumerate}
    \item Разбиваем сектор $A_0 O A_n$ лучами $\alpha = \varphi_0 < \varphi_1 < \ldots < \varphi_n = \beta$ на углы $\angle A_0 O A_1$, $\angle A_1 O A_2$, $\ldots$, $\angle A_{n-1} O A_n$\\
    $\Delta \varphi_i = \varphi_i - \varphi_{i-1}$ --- величина $\angle A_{i-1} O A_i$ в радианах\\
    $\lambda = \underset{i}{\max} \Delta \varphi_i,\ i = \overline{1, n}$ % рисунок
    \item $\forall$ выберем и проведём $\Psi_i$, $\Psi_i \in \angle A_{i-1} O A_i$\\
    Находим $r = r(\Psi_i)$\\
    $M_i \big(\Psi_i,\ r(\Psi_i)\big)$,\quad $M_i \in \angle A_{i-1} O A_i$,\quad $M_i \in r = r(\varphi)$ % рисунок
    \item Заменяем каждый $i$-ый криволинейный сектор на круговой сектор $R = r(\Psi_i),\ i = \overline{1, n}$ % рисунок -> рисунок
    $\displaystyle S_i = \frac{1}{2} R^2\cdot \Delta \varphi_i$ --- площадь $i$-го кругового сектора\\
    $R = r(\Psi_i)$ \\[1ex]
    $\displaystyle \sum_{i=1}^{n} S_i = \sum_{i=1}^{n} \frac{1}{2} r^2 (\Psi_i) \cdot \Delta \varphi_i = \frac{1}{2} \sum_{i=1}^{n} r^2(\Psi_i)\cdot \Delta \varphi_i$
    \item Вычисляем предел
    \begin{flalign*}
        & \lim_{\lambda \to 0} \frac{1}{2} \sum_{i=1}^{n} r^2 (\Psi_i)\cdot \Delta \varphi_i = \boxed{\frac{1}{2} \int_{\alpha}^{\beta} r^2 \dd{\varphi} = S} &
    \end{flalign*}
\end{enumerate}

\subsection{Вывести формулу для вычисления длины дуги графика функции $y = f(x)$, отсечённой прямыми $x = a$ и $x = b$}

Пусть $y=f(x)$ непрерывна на $[a;b]$.\\
% рисунок
$M_0 (x_0, y_0)\quad M(x, y)$\\
$\Delta x$ --- приращение $x$\quad $\Delta y$ --- приращение $y$\\[1ex]
$\begin{aligned}
    x &\to x + \Delta x\\
    y &\to y + \Delta y
\end{aligned}\quad M(x, y) \to M_1(x + \Delta x, y + \Delta y)$\\[1ex]
$l_0$ --- $\wideparen{M_0M}$ --- дуга кривой\quad $\Delta l$ --- приращение дуги кривой\quad $\Delta l = \wideparen{MM_1}$\\[1ex]
Найдём $l'_x$ --- ?\\
$\displaystyle l'_x = \lim_{\Delta x \to 0} \frac{\Delta l}{\Delta x} $\\[1ex]
$\triangle MM_1A\quad MA = \Delta x\quad AM_1 = \Delta y $
\begin{flalign*}
    & MM_1^2 = \Delta x ^2 + \Delta y^2\quad |\cdot \Delta l^2\ | : \Delta l^2 &\\
    & \left(\frac{MM_1}{\Delta l}\right)^2 \cdot (\Delta l)^2 = \Delta x^2 + \Delta y^2\quad | : \Delta x^2 &\\
    & \left(\frac{MM_1}{\Delta l}\right)^2 \cdot \left(\frac{\Delta l}{\Delta x}\right)^2 = 1 + \left(\frac{\Delta y}{\Delta x}\right)^2 &
\end{flalign*}
Вычислим предел при $\Delta x \to 0$.\\
Левая часть:
\begin{flalign*}
    & \lim\limits_{\Delta x \to 0} \left(\frac{MM_1}{\Delta l}\right)^2 \cdot \left(\frac{\Delta l}{\Delta x}\right)^2 = \left| \begin{aligned}
        &\text{при } \Delta x \to 0\quad M\to M_1 \\
        &\Delta l \to MM_1\quad \text{дуга } \to \text{ хорде}
    \end{aligned} \right| = \lim\limits_{\Delta x \to 0} 1^2 \cdot \left(\frac{\Delta l}{\Delta x}\right)^2 = (l'_x)^2 &
\end{flalign*}
Правая часть:
\begin{flalign*}
    & \lim\limits_{\Delta x \to 0}  \left(1 + \left(\frac{\Delta y}{\Delta x}\right)^2\right) = 1 + \lim\limits_{\Delta x \to 0} \left(\frac{\Delta y}{\Delta x}\right)^2 = 1 + (y'_x)^2 &
\end{flalign*}
Получаем:
\begin{align*}
    (l'_x)^2 &= 1 + (y'_x)^2 \\
    l'_x &= \sqrt{1 + (y'_x)^2}\quad |\cdot \dd{x}\\
    l'_x\dd{x} &= \sqrt{1 + (y'_x)^2}\dd{x}\\
    \dd{l} &= \sqrt{1 + (y'_x)^2}\dd{x} \tag{$\vee$} 
\end{align*}
\begin{gather}
    \boxed{l = \int_{a}^{b} \sqrt{1 + (y'_x)^2} \dd{x}}
\end{gather}

\newpage
\section{Используемые теоремы}

\begin{theorem}\hlabel{т: опр. интеграл и константа}
    Если $C$ --- $const$, то 
    \begin{gather*}
        \boxed{\int_{a}^{b} C\, dx = C\cdot (b-a)}
    \end{gather*}
\end{theorem}

\begin{theorem}\hlabel{т: линейная комбинация интегрируемых}
    Если функции $f_1(x),\ f_2(x)$ интегрируемы на $[a;b]$, то их линейная комбинация
    \begin{gather*}
        \lambda_1 f_1(x) + \lambda_2 f_2(x),\ \text{где } \lambda_1,\ \lambda_2 \in \R
    \end{gather*}
    интегрируема на $[a;b]$ и верно равенство:
    \begin{gather*}
        \int_{a}^{b}\Big(\lambda_1 f_1(x) + \lambda_2 f_2(x)\Big)\, dx = \lambda_1 \int_{a}^{b} f_1(x)\, dx + \lambda_2 \int_{a}^{b} f_2(x)\, dx
    \end{gather*}
\end{theorem}

\begin{theorem}[Об интегрировании неравенства]\hlabel{т: интегрирование неравенства}
    Пусть функции $f(x)$ и $g(x)$ интегрируемы на $[a;b]$ и $\forall x \in [a;b]\colon f(x) \geqslant g(x)$, то
    \begin{gather*}
        \boxed{\int_{a}^{b} f(x)\, dx \geqslant \int_{a}^{b}g(x)\, dx}
    \end{gather*}
\end{theorem}

\begin{theorem}[Непрерывность $I(x)$]\hlabel{т: непрерывность I(x)}
    Если функция $f(x)$ на $[a;b]$ непрерывна, то $I(x) = \int_{a}^{x} f(t)\, dt$ --- непрерывна на $[a;b]$.
\end{theorem}

\end{document}