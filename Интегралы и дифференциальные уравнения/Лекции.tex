\input{../../preamble2.tex}

\begin{document}

\begin{titlepage}
    \vspace*{0pt}
    \vfill
    \centering
    \Huge\textbf{Интегралы и дифференциальные уравнения} \\[7pt]
    \Large\textbf{Лекции} \\
    \large 2 семестр \\ 
    \vfill
    \begin{flushright}
        \normalsize GitHub: \href{https://github.com/malyinik}{malyinik} \\
    \end{flushright}
    \normalsize 2024 г.
\end{titlepage}
\newpage

\tableofcontents
\newpage
\input{Первообразная и неопределённый интеграл.tex}
\newpage
\zerocounter
\input{Правильные и неправильные рациональные дроби.tex}
\zerocounter
\newpage
\input{Определённый интеграл и криволинейная трапеция.tex}
\zerocounter
\newpage


%%%%%%%%

\zerocounter
\subsection{Методы вычисления определённого интеграла}

\subsubsection{Метод интегрирования по частям}

\begin{theorem}
    Пусть функции $u = u(x)$ и $\upsilon = \upsilon(x)$ непрерывно дифференцируемы на $[a;b]$. Тогда имеет место равенство
    \begin{gather*}
        \boxed{\int_{a}^{b} u\, d\upsilon = u\eval{\upsilon}_{a}^{b} - \int_{a}^{b} \upsilon\, du}
    \end{gather*}
\end{theorem}
\begin{proof}
    Рассмотрим произведение функций $u\cdot \upsilon$. \\
    Дифференцируем: \vspace{-\topsep}
    \begin{align*}
        d(u\cdot \upsilon) &=  \upsilon\, du + u\, d\upsilon \\ 
        u\, d\upsilon &= d(u \upsilon) - \upsilon\, du
    \end{align*}
    Интегрируем:
    \begin{align*}
        \int_{a}^{b} u\, d\upsilon = \int_{a}^{b} \big(d(u\upsilon) - \upsilon\, du\big) = \int_{a}^{b} d(u\upsilon) - \int_{a}^{b} \upsilon\, du - u\eval{\upsilon}_{a}^{b} - \int_{a}^{b} \upsilon\, du
    \end{align*}
\end{proof}
\begin{eg}
    \begin{flalign*}
        & \int_{1}^{e} \ln x \dd{x} = \left| \begin{aligned}
            & u = \ln x\quad \dd{u} = \frac{1}{x}\dd{x} \\
            & \dd{\upsilon} = \dd{x}\quad \upsilon = x
        \end{aligned}\right| = x\eval{\ln x}_{1}^{e} - \int_{1}^{e}x \cdot \frac{1}{x}\dd{x} = (e \ln e - 1\cdot \ln 1) - \eval{x}_{1}^{e} = & \\
        & = e - (e - 1) = \cancel{e} - \cancel{e} + 1 = 1 &
    \end{flalign*}
\end{eg}


%%%%%%

\end{document}