\input{../../preamble2.tex}

\begin{document}

\begin{titlepage}
    \vspace*{0pt}
    \vfill
    \centering
    \Huge\textbf{Интегралы и дифференциальные уравнения} \\[7pt]
    \Large\textbf{Лекции} \\
    \large 2 семестр \\ 
    \vfill
    \begin{flushright}
        \normalsize GitHub: \href{https://github.com/malyinik}{malyinik} \\
    \end{flushright}
    \normalsize 2024 г.
\end{titlepage}
\newpage

\tableofcontents
\newpage
\input{Первообразная и неопределённый интеграл.tex}
\newpage
\zerocounter
\input{Правильные и неправильные рациональные дроби.tex}
\zerocounter
\newpage


%%%%%%%%

\section{Определённый интеграл. Криволинейная трапеция}

\subsection{Определённый интеграл}

\subsubsection*{Пусть функция $y = f(x)$ определена на $[a;b]$}

\begin{definition}
    Множество точек $a = x_0 < x_1 < \ldots < x_i < \ldots < x_n = b$ называется \textbf{разбиением отрезка} $\bm{[a;b]}$, при этом отрезки $[x_{i-1}; x_i]$ называются \textbf{отрезками разбиения}. \\[2ex]
    $i = 1,\ldots,n\quad i = \overline{1, n}$\\
    $\Delta x_i = x_i - x_{i-1}$ --- длина $i$-го отрезка разбиения\quad $i = \overline{1, n}$\\
    $\lambda = \underset{i}{\max}\, \Delta x_i$ --- диаметр разбиения\\
\end{definition}
Рассмотрим произвольное разбиение $[a;b]$. В каждом из отрезков разбиения $[x_{i-1}; x_i]$ выберем точку $\xi_i,\ i = \overline{1, n}$. Составим сумму
\begin{gather}\hlabel{ф: интегральная сумма}
    \boxed{S_n = \sum_{i=1}^{n}f(\xi_i)\cdot \Delta x_i}
\end{gather}
(\ref{ф: интегральная сумма}) --- интегральная сумма для функции $y=f(x)$ на $[a;b]$.

% рисунок

\begin{figure}[h]
    \centering
    \begin{tikzpicture}
        \tkzInit[xmin=-0.5, xmax=6, ymin=-0.5, ymax=2]
        \tkzDrawX \tkzDrawY
        \tkzDefPoint(3, 1.45){f}
        \tkzDrawPoint[fill=black, size = 3pt](f)
        \node[below left] at (0, 0) {$0$};
        \draw[very thick] (1, 1) .. controls (2, 2.5) and (4, 0.5) .. (5, 1.5) node[right]{$y=f(x)$};
        \draw[thick, dotted] (1, 0) node[below]{$\underset{x_0}{a}$} -- (1, 1);
        \draw[thick, dotted] (1.5, 0) node[below]{$x_1$} -- (1.5, 1.45);
        \draw[thick, dotted] (2, 0) -- (2, 1.6);
        \draw[thick, dotted] (2.5, 0) node[below]{\scriptsize{$x_{i-1}$}} -- (2.5, 1.5);
        \draw[thick, dotted] (3, 0) node[below]{$\xi_i$} -- (3, 1.45) node[above]{$f(\xi_i)$};
        \draw[thick, dotted] (3.5, 0) node[below]{$x_i$} -- (3.5, 1.3);
        \draw[thick, dotted] (4, 0) -- (4, 1.2);
        \draw[thick, dotted] (4.5, 0) -- (4.5, 1.25);
        \draw[thick, dotted] (5, 0) node[below]{$\underset{x_n}{b}$} -- (5, 1.5);
    \end{tikzpicture}
\end{figure}

\begin{definition}\hlabel{опр: определённый интеграл}
    \textbf{Определённым интегралом} от функции $y=f(x)$ на $[a;b]$ называется \underline{конечный} предел интегральной суммы (\ref{ф: интегральная сумма}), когда число отрезков разбиения растёт, а их длины стремятся к нулю.
    \begin{gather}\hlabel{ф: определённый интеграл}
        \boxed{\int_{a}^{b} f(x)\, dx = \lim_{\lambda \to 0} \sum_{i=1}^{n} f(\xi_i)\cdot \Delta x}
    \end{gather}
    Предел (\ref{опр: неопределённый интеграл}) не зависит от способа разбиения отрезка $[a;b]$ и выбора точек $\xi_i,\ \overline{1, n}$.\\
    $f(x)$ --- подынтегральная функция\\
    $f(x)\, dx$ --- подынтегральное выражение\\
    $\displaystyle\int_{a}^{b}$ --- знак определённого интеграла\\
    $a$ --- нижний предел интегрирования\\
    $b$ --- верхний предел интегрирования
\end{definition}

\newpage
\subsection{Криволинейная трапеция}

\begin{definition}
    \textbf{Криволинейной трапецией} называется фигура, ограниченная графиком функции $y=f(x)$, отрезком $[a;b]$ на $Ox$, прямыми $x = a$ и $x = b$ параллельными оси $Oy$.
\end{definition}

\subsubsection{Геометрический смысл}

% рисунок 

\begin{figure}[h]
    \centering
    \begin{tikzpicture}
        \tkzInit[xmin=-0.5, xmax=6, ymin=-0.5, ymax=2]
        \tkzDrawX \tkzDrawY
        \node[below left] at (0, 0) {$0$};
        \draw[very thick] (1, 1) .. controls (2, 2.5) and (4, 0.5) .. (5, 1.5) node[right]{$y=f(x)$};
        \draw[thick] (1, 0) -- (1, 1);
        \draw[thick] (5, 0) -- (5, 1.5);

        \filldraw[pattern=north east lines, pattern color=gray] (1, 0) -- (1, 1) .. controls (2, 2.5) and (4, 0.5) .. (5, 1.5) -- (5, 0) -- cycle;

        \node at (3, 0.5) {$\bm S_{\textbf{кр. тр.}}$};
    \end{tikzpicture}
\end{figure}

Определённый интеграл численно равен площади криволинейной трапеции.
\begin{gather*}
    S_{\text{кр. тр.}} = \int_{a}^{b} f(x)\, dx
\end{gather*}

\begin{definition}
    Функция $y=f(x)$ называется \textbf{интегрируемой} на $[a;b]$, если существует конечный предел интегральной суммы (\ref{ф: интегральная сумма}) на $[a;b]$.
\end{definition}

\begin{theorem}[Существование определённого интеграла]
    Если функция $y=f(x)$ непрерывна на $[a;b]$, то она на этом отрезке интегрируема.
\end{theorem}

\subsection{Свойства определённого интеграла}

\begin{theorem}
    Если функция $y=f(x)$ интегрируема на отрезке $[a;b]$, то имеет место равенство
    \begin{gather*}
        \boxed{\int_{a}^{b} f(x) dx = - \int_{b}^{a} f(x) dx}
    \end{gather*}
\end{theorem}
\begin{proof}
    По определению определённого интеграла (\textbf{опр. \ref{опр: определённый интеграл}})
    \begin{align*}
        \int_{a}^{b} f(x)\, dx = \lim_{\lambda \to 0} \sum_{i=1}^{n} f(\xi_i)\cdot \Delta x = \lim_{\lambda \to 0} \sum_{i=1}^{n} f(\xi_i)(x_i - x_{i-1}) &= - \lim_{\lambda \to 0} \sum_{i=1}^{n} f(\xi_i) (x_{i-1} - x_i) \\
        & = - \int_{b}^{a} f(x)\, dx
    \end{align*}
\end{proof}

\newpage
\begin{theorem}[Аддитивность определённого интеграла]
    Если функция $y=f(x)$ интегрируема на каждом из отрезков $[a;c],\ [c;b]\ (a < c < b)$, то она интегрируема на $[a;b]$ и верно равенство
    \begin{gather*}
        \boxed{\int_{a}^{b} f(x)\, dx = \int_{a}^{c} f(x)\, dx + \int_{c}^{b} f(x)\, dx}
    \end{gather*}
\end{theorem}
\begin{proof}
    Рассмотрим произвольное разбиение $[a;b]$ такое, что одна из точек разбиения совпадает с точкой $c$:
    \begin{gather*}
        a = x_0 < x_1 < \ldots < x_m = c < x_{m+1} < \ldots < x_n = b
    \end{gather*}
    Данное разбиение определяет ещё два разбиения:
    \begin{gather*}
        \begin{aligned}
            a = x_0 < x_1 &< \ldots < x_{m-1} < x_m = c\\
            c = x_m < x_{m+1} &< \ldots < x_{n-1} < x_n = b
        \end{aligned}\qquad
        \begin{aligned}
            &\lambda_1 = \underset{i}{\max}\, \Delta x_i,\ i = \overline{1, m} \\
            &\lambda_2 = \underset{i}{\max}\, \Delta x_i,\ i = \overline{m+1, n}
        \end{aligned}
    \end{gather*}
    Так как функция $y=f(x)$ интегрируема на $[a;c]$ и на $[c;b]$, то
    \begin{gather*}
        \int_{a}^{c} f(x)\, dx = \lim_{\lambda_1 \to 0} \sum_{i = 1}^{m} f(\xi_i) \cdot \Delta x_i\\
        \int_{c}^{b} f(x)\, dx = \lim_{\lambda_2 \to 0} \sum_{i = m + 1}^{n} f(\xi_i) \cdot \Delta x_i
    \end{gather*}
    $\lambda = \max \{\lambda_1; \lambda_2\}\quad \lambda \to 0$\\
    Суммируем интегральные суммы:
    \begin{gather*}
        \sum_{i = 1}^{m} f(\xi_i) \cdot \Delta x_i + \sum_{i = m + 1}^{n} f(\xi_i) \cdot \Delta x_i = \sum_{i=1}^{n} f(\xi_i)\cdot \Delta x_i
    \end{gather*}
    Вычислим предел:
    \begin{align*}
        &\lim_{\lambda \to 0} \left( \sum_{i=1}^{m} f(\xi_i)\cdot \Delta x_i + \sum_{i=m+1}^{n} f(\xi_i) \cdot \Delta x_i \right) = \lim_{\lambda \to 0} \sum_{i=1}^{n} f(\xi_i) \cdot \Delta x_i \\
        &\lim_{\lambda \to 0} \sum_{i=1}^{m} f(\xi_i) \cdot \Delta x_i + \lim_{\lambda\to 0} \sum_{i = m+1}^{n} f(\xi_i)\cdot \Delta x_i = \lim_{\lambda \to 0}\sum_{i=1}^{n} f(\xi_i)\cdot \Delta x_i \\
        &\lim_{\lambda_1 \to 0} \sum_{i=1}^{m} f(\xi_i) \cdot \Delta x_i + \lim_{\lambda_2 \to 0} \sum_{i = m+1}^{n} f(\xi_i)\cdot \Delta x_i = \lim_{\lambda \to 0}\sum_{i=1}^{n} f(\xi_i)\cdot \Delta x_i
    \end{align*}
    Из последнего равенства следует, что $f(x)$ --- интегрируема на $[a;b]$ и верно равенство
    \begin{gather*}
        \int_{a}^{c} f(x)\, dx + \int_{c}^{b} f(x)\, dx = \int_{a}^{b} f(x)\, dx
    \end{gather*}
\end{proof}

\begin{theorem}
    Если $C$ --- $const$, то 
    \begin{gather*}
        \boxed{\int_{a}^{b} c\, dx = c\cdot (b-a)}
    \end{gather*}
\end{theorem}
\begin{proof}
    \begin{gather*}
        \int_{a}^{b} c\, dx = \lim_{\lambda \to 0} \sum_{i=1}^{n} c \cdot \Delta x_i = c \cdot \lim_{\lambda \to 0} \sum_{i=1}^{n} x_i - x_{i-1} = c\cdot (b - a)
    \end{gather*}    
\end{proof}

\begin{theorem}
    Если функции $f_1(x),\ f_2(x)$ интегрируемы на $[a;b]$, то их линейная комбинация
    \begin{gather*}
        \lambda_1 f_1(x) + \lambda_2 f_2(x),\ \text{где } \lambda_1,\ \lambda_2 \in \R
    \end{gather*}
    интегрируема на $[a;b]$ и верно равенство:
    \begin{gather*}
        \int_{a}^{b}\Big(\lambda_1 f_1(x) + \lambda_2 f_2(x)\Big)\, dx = \lambda_1 \int_{a}^{b} f_1(x)\, dx + \lambda_2 \cdot \int_{a}^{b} f_2(x)\, dx
    \end{gather*}
\end{theorem}

\begin{proof}
    \begin{align*}
        \int_{a}^{b}\Big(\lambda_1 f_1(x) + \lambda_2 f_2(x)\Big)\, dx &= \lim_{\lambda \to 0} \sum_{i=1}^{n} \Big(\lambda_1 f_1(\xi_i) + \lambda_2 f_2(\xi_i)\Big)\cdot \Delta x_i = \\
        &= \lim_{\lambda \to 0} \sum_{i=1}^{n} \Big(\lambda_1 f_1(\xi_i) \cdot \Delta x_i + \lambda_2 f_2(\xi_i)\cdot \Delta x_i\Big) = \\
        &= \lim_{\lambda \to 0} \left(\sum_{i=1}^{n} \lambda_1 f_1(\xi_i) \cdot \Delta x_i + \sum_{i=1}^{n} \lambda_2 f_2(\xi_i)\cdot \Delta x_i\right) = \\
        &= \lim_{\lambda \to 0} \sum_{i=1}^{n} \lambda_1 f_1(\xi_i) \cdot \Delta x_i + \lim_{\lambda \to 0} \sum_{i=1}^{n} \lambda_2 f_2(\xi_i)\cdot \Delta x_i = \\
        &= \lambda_1 \lim_{\lambda \to 0} \sum_{i=1}^{n} f_1(\xi_i) \cdot \Delta x_i + \lambda_2 \lim_{\lambda \to 0} \sum_{i=1}^{n} f_2(\xi_i)\cdot \Delta x_i = \\
        & = \lambda_1 \int_{a}^{b} f_1(x)\, dx + \lambda_2 \int_{a}^{b} f_2(x)\, dx = \\
        & = \int_{a}^{b}\Big(\lambda_1 f_1(x) + \lambda_2 f_2(x)\Big)\, dx
    \end{align*}
\end{proof}

\newpage
\begin{corollary}
    \begin{gather*}
        \int_{a}^{a} f(x)\, dx = 0
    \end{gather*}
\end{corollary}

\begin{theorem}[О сохранении определённым интегралом знака подынтегральной функции]
    Если $f(x)$ интегрируема и неотрицательна на $[a;b]$, то 
    \begin{gather*}
        \boxed{\int_{a}^{b} f(x)\, dx \geqslant 0}
    \end{gather*}
\end{theorem}
\begin{proof}
    \begin{gather*}
        \int_{a}^{b} f(x)\, dx = \lim_{\lambda\to 0} \sum_{i=1}^{n} f(\xi_i)\cdot \Delta x_i
    \end{gather*}
    $\Delta x_i$ --- длины отрезков разбиения\qquad $\Delta x_i > 0$ \\
    $f(\xi_i) \geqslant 0$\quad по условию
    \begin{gather*}
        f(\xi_i)\cdot \Delta x_i \geqslant 0,\ i = \overline{i, n}\\
        \sum_{i=1}^{n} f(\xi_i)\cdot \Delta x_i \geqslant 0\quad \text{как сумма неотрицательных чисел}\\
        \lim_{\lambda \to 0} \sum_{i=1}^{n} f(\xi_i) \cdot \Delta x_i \geqslant 0\quad \begin{aligned} &\text{по следствию из теоремы } \textit{о сохранении} \\ &\textit{функцией знака своего предела} \end{aligned}\\
        \Downarrow\\
        \int_{a}^{b} f(x)\, dx \geqslant 0
    \end{gather*}
\end{proof}

%%%%%%
\end{document}