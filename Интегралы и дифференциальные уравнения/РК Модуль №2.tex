\input{../../preamble2.tex}

\begin{document}

\begin{titlepage}
    \vspace*{0pt}
    \vfill
    \centering
    \Huge\textbf{Интегралы и дифференциальные уравнения} \\[7pt]
    \Large\textbf{Рубежный контроль} \\
    \large 2 семестр | Модуль №2
    \vfill
    \begin{flushright}
        \normalsize GitHub: \href{https://github.com/malyinik}{malyinik} \\
    \end{flushright}
    \normalsize 2024 г.
\end{titlepage}
\newpage

\tableofcontents
\newpage

\section{Теоретические вопросы, оцениваемые в 1 балл}

\subsection{Сформулировать определение общего решения ОДУ n-го порядка}
\vspace{-2\topsep}
\begin{gather*}
    \left\{ \begin{aligned}
        y(x_0) &= y_0 \\
        y'(x_0) &= y_{10} \\
        y''(x_0) &= y_{20} \\
        \ldots\ldots\ldots&\ldots\ldots \\
        y^{(n-1)}(x_0) &= y_{n\cdot 10} \\
    \end{aligned} \right. \quad \text{--- начальное условие}
\end{gather*}

\begin{definition*}
    \textbf{Общим решением} ДУ n-го порядка называется функция \break$y = \varphi(x,\, C_1,\, C_2,\, \ldots,\, C_n)$ удовлетворяющая условиям:
    \begin{enumerate}
        \item $y = \varphi(x,\, C_1,\, C_2,\, \ldots,\, C_n)$ --- решение ДУ n-го порядка при любых $C_1, C_2, \ldots, C_n$\nobreak --- $const$
        \item Какого бы ни было начальное условие, можно найти точки $C_1^0, C_2^0, \ldots, C_n^0$, что функция $y = \varphi(x,\, C_1^0,\, C_2^0,\, \ldots,\, C_n^0)$ будет удовлетворять начальным условиям.
    \end{enumerate}
\end{definition*}

\subsection{Сформулировать определение задачи Коши для ОДУ n-го порядка}

\begin{definition*}
    \textbf{Задача Коши} для ДУ n-го порядка заключается в отыскании решения ДУ, удовлетворяющего начальному условию:
    \begin{gather*}
        \left\{ \begin{aligned}
            y(x_0) &= y_0 \\
            y'(x_0) &= y_{10} \\
            y''(x_0) &= y_{20} \\
            \ldots\ldots\ldots \\
            y^{(n-1)}(x_0) &= y_{n\cdot 10} \\
        \end{aligned} \right.
    \end{gather*}
    \[
        \text{Задача Коши} = \text{ДУ} + \text{начальное условие}
    \]
\end{definition*}

\subsection{Сформулировать определение линейного ОДУ n-го порядка}

\begin{definition*}
    ДУ n-го порядка называется \textbf{линейным}, если неизвестная функция $y(x)$ и её производные до n-го порядка включительно входят в уравнение в первой степени, не перемножаясь между собой.
    \begin{gather}
        \boxed{y^{(n)} + p_1(x)y^{(n-1)}  + p_2(x)y^{(n-2)} + \ldots + p_{n-1}(x)y' + p_n(x)y = f(x)}
    \end{gather}
    Если $f(x) = 0,\ \forall x \in I$, то ДУ (1) называется \textbf{линейным однородным ДУ} (ЛОДУ).
    \begin{gather}
        \boxed{y^{(n)} + p_1(x)y^{(n-1)}  + p_2(x)y^{(n-2)} + \ldots + p_{n-1}(x)y' + p_n(x)y = 0}
    \end{gather}
    (2) ЛОДУ n-го порядка
\end{definition*}

\subsection{Сформулировать определение линейной зависимости и линейной независимости системы функций на промежутке}

\begin{definition*}
    Система функций $y_1(x),\ \ldots,\ y_n(x)$ называется \textbf{линейно зависимой} на некотором промежутке $I$, если их линейная комбинация равна нулю, то есть 
    \[
        C_1y_1(x) + \ldots + C_n y_n = 0
    \]
    при этом существует хотя бы один $C_i \ne 0,\ i = \overline{1,n},\ C_1,\ \ldots,\ C_n$ --- $const$
\end{definition*}

\begin{definition*}
    Система функций $y_1(x),\ \ldots,\ y_n(x)$ называется \textbf{линейно независимой} на некотором промежутке $I$, если их линейная комбинация равна нулю, то есть 
    \[
        C_1y_1(x) + \ldots + C_n y_n = 0
    \]
    где все $C_1 = 0,\ i = \overline{1,n}$
\end{definition*}

\subsection{Сформулировать определение определителя Вронского системы функций}

\begin{definition*}
    \textbf{Определителем Вронского} (вронскианом) системы $(n - 1)$ раз дифференцируемых функций $y_1 (x),\ \ldots,\ y_n(x)$ называется определитель вида:
    \begin{gather*}
        W (x) = \begin{vmatrix}
            y_1(x) & y_2(x) & \cdots & y_n(x) \\
            y_1'(x) & y_2'(x) & \cdots & y_n'(x) \\
            \cdots & \cdots & \cdots & \cdots \\
            y_1^{(n-1)}(x) & y_2^{(n-1)}(x) & \cdots & y_n^{(n-1)}(x) \\
        \end{vmatrix}
    \end{gather*}
\end{definition*}

\setcounter{equation}{0}
\subsection{Сформулировать определение фундаментальной системы решений линейного однородного ОДУ}

Пусть дано ЛОДУ n-го порядка
\begin{align}
    y^{(n)} + p_1(x) y^{(n-1)} + \ldots + p_{n-1}(x)y' + p_n(x)y = 0
\end{align}

\begin{definition*}
    \textbf{Фундаментальной системой решений ЛОДУ n-го порядка} (1) называется любая система \underline{линейно независимых} частных решений ЛОДУ n-го порядка.
\end{definition*}

\newpage
\subsection{Сформулировать определение характеристического уравнения линейного ОДУ с постоянными коэффициентами}

\begin{align*}
    y^{(n)} + a_1 y^{(n-1)} + \ldots + a_{n-1} y' + a_n y = 0 \tag{1} \\
    \boxed{k^n + a_1k^{n-1} + \ldots + a_{n-1} k + a_n = 0} \tag{2}
\end{align*}
\begin{definition}
    Уравнение (2) называется характеристическим уравнением. \textbf{Характеристическое уравнение} --- это алгебраическое уравнение/полином/многочлен, полученный из ДУ (1) путём замены n-ой производной неизвестной функции $y$ на n-ую степень величины $k$, а сама функция $y$ заменена на единицу.
\end{definition}

\section{Теоретические вопросы, оцениваемые в 3 балла}

\subsection{Сформулировать и доказать теорему о вронскиане системы линейно зависимых функций}

\begin{theorem*}[О вронскиане линейно зависимых функций]
    Если $(n-1)$ раз дифференцируемые функции $y_1(x),\ \ldots,\ y_n(x)$ линейно зависимы на некотором промежутке $I$, то
    \[
        W(x) = 0,\ \forall x \in I
    \]
\end{theorem*}
\begin{proof}
    Так как $y_1(x),\ \ldots,\ y_n(x)$ линейно зависимы на $I$, то
    \begin{gather*}
        \boxed{C_1 y_1(x),\ \ldots,\ C_n y_n(x) = 0}\qquad \exists\, C_i \ne 0,\ i = \overline{1,n} \tag{$*$}
    \end{gather*}
    Продифференцируем $(*)\ (n-1)$ раз:
    \begin{gather*}
        \boxed{C_1 y_1'(x),\ \ldots,\ C_n y_n'(x) = 0}\qquad \exists\, C_i \ne 0,\ i = \overline{1,n} \tag{$**$}
    \end{gather*}
    По определению линейной зависимости $\Rightarrow$ $y_1'(x),\ \ldots,\ y_n'(x)$ --- линейно зависимы
    \begin{gather*}
        \cdots\cdots\cdots\cdots\cdots\cdots\cdots\cdots\cdots\cdots \\
        \boxed{C_1 y_1^{(n-1)}(x),\ \ldots,\ C_n y_n^{(n-1)}(x) = 0}\qquad \exists\, C_i \ne 0,\ i = \overline{1,n} \tag{$***$}
    \end{gather*}    
    По определению линейной зависимости $\Rightarrow$ $y_1^{(n-1)}(x),\ \ldots,\ y_n^{(n-1)}(x)$ --- линейно зависимы \\
    Составим систему из $(*),\ (**)$ и $(***)$:
    \begin{gather*}
        \left\{ \begin{aligned}
            C_1 y_1(x),\, \ldots,\, C_n y_n(x) &= 0 \\
            C_1 y_1'(x),\, \ldots,\, C_n y_n'(x) &= 0 \\
            \cdots\cdots\cdots\cdots\cdots\cdots\cdots&\cdots\cdot \\
            C_1 y_1^{(n-1)}(x),\, \ldots,\, C_n y_n^{(n-1)}(x) &= 0
        \end{aligned} \right.
    \end{gather*}
    Это однородная СЛАУ относительно $C_1,\, \ldots,\, C_n$. \\
    Определитель этой СЛАУ:
    \begin{gather*}
        \begin{vmatrix}
            y_1(x) & \cdots & y_n(x) \\
            y_1'(x) & \cdots & y_n'(x) \\
            \cdots & \cdots & \cdots \\
            y_1^{(n-1)}(x) & \cdots & y_n^{(n-1)}(x) \\
        \end{vmatrix} = W(x) \text{ --- это определитель Вронского}
    \end{gather*}
    $W(x) = 0$, так как все строки определителя линейно зависимы.
\end{proof}

\subsection{Сформулировать и доказать теорему о вронскиане системы линейно независимых частных решений линейного однородного ДУ}

\begin{theorem*}[О вронскиане линейно независимых частных решений ЛОДУ n-го порядка]
    Если функции $y_1(x),\, \ldots,\, y_n(x)$ линейно независимы  на некотором промежутке $I$ и являются частными решениями ЛОДУ n-го порядка
    \[
        y^{(n)} + p_1(x)y^{(n-1)} + \ldots + p_{n-1}(x) y' + p_n(x) y = 0
    \]
    с непрерывными на промежутке $I$ коэффициентами $p_1(x),\, \ldots,\, p_n(x)$, то 
    \[
        W(x) \ne 0,\ \forall x \in I
    \]
\end{theorem*}
\begin{proof}[Метод от противного]
    Предположим, что $\exists\, x_0 \in I\colon W(x_0) \ne 0$
    \begin{gather*}
        W (x_0) = \begin{vmatrix}
            y_1(x_0) & y_2(x_0) & \cdots & y_n(x_0) \\
            y_1'(x_0) & y_2'(x_0) & \cdots & y_n'(x_0) \\
            \cdots & \cdots & \cdots & \cdots \\
            y_1^{(n-1)}(x_0) & y_2^{(n-1)}(x_0) & \cdots & y_n^{(n-1)}(x_0)\\
        \end{vmatrix} = 0
    \end{gather*}
    Построим СЛАУ по определителю
    \begin{gather*}
        \left\{ \begin{aligned}
            C_1 y_1(x_0) + C_2 y_2(x_0) + \ldots + C_n y_n(x_0) &= 0\\
            C_1 y_1'(x_0) + C_2 y_2'(x_0) + \ldots + C_n y_n'(x_0) &= 0\\
            \cdots\cdots\cdots\cdots\cdots\cdots\cdots\cdots\cdots\cdots\cdots\cdots&\cdots\cdot \\
            C_1 y_1^{(n-1)}(x_0) + C_2 y_2^{(n-1)}(x_0) + \ldots + C_n y_n^{(n-1)}(x_0) &= 0 \\
        \end{aligned} \right.
    \end{gather*}
    Данная СЛАУ имеет ненулевое решение, так как $W(x_0) = 0$\\
    Рассмотрим функцию 
    \[
    y = C_1y_1 + \ldots + C_ny_n
    \]
    Так как $y_1,\ \ldots,\ y_n$ --- частные решения ЛОДУ n-го порядка, то по теореме \textit{о частных решениях ЛОДУ}:
    \[
        y = C_1y_1 + \ldots + C_ny_n \text{ --- решение ЛОДУ n-го порядка}
    \]
    Найдём $y(x_0)$:
    \[
        y(x_0) = C_1y_1(x_0) + \ldots + C_ny_n(x_0) = 0
    \]
    Дифференцируем $(n-1)$ раз функцию $y = C_1y_1 + \ldots + C_ny_n$:
    \begin{align*}
        y'(x_0) &= C_1y_1'(x_0) + \ldots + C_ny_n'(x_0) = 0 \\
        y''(x_0) &= C_1y_1''(x_0) + \ldots + C_ny_n''(x_0) = 0 \\
        \cdots\cdots&\cdots\cdots\cdots\cdots\cdots\cdots\cdots\cdots\cdots\cdots\cdots \\
        y^{(n-1)}(x_0) &= C_1y_1^{(n-1)}(x_0) + \ldots + C_ny_n^{(n-1)}(x_0) = 0
    \end{align*}
    Получили, что $y = C_1y_1 + \ldots + C_ny_n$ --- решение ЛОДУ n-го порядка (2), удовлетворяющее начальному условию:
    \begin{gather*}
        \left\{ \begin{aligned}
            y(x_0) &= 0 \\
            y'(x_0) &= 0 \\
            \cdots\cdots&\cdots \\
            y^{(n-1)}(x_0) &= 0 \\
        \end{aligned}\right.
    \end{gather*}
    Но $y=0$ -- решение ЛОДУ (2), удовлетворяющее начальному условию (3) \\
    По теореме $\exists!$ \textit{решения задачи Коши для линейного дифференциального уравнения n-го порядка} $\Rightarrow$
    \[
        y = C_1y_1 + \ldots + C_ny_n = 0
    \]
    при этом $C_1,\, \ldots,\, C_n$ -- ненулевые константы $\Rightarrow$ $y_1,\, \ldots,\, y_n$ --- линейно зависимы по определению линейной зависимости.
    Это противоречит условию $\Rightarrow$ предположение не является верным $\forall x \in I\colon W(x) \ne 0$
\end{proof}

\subsection{Сформулировать и доказать теорему о существовании фундаментальной системы решений линейного однородного ДУ n-го порядка}

\begin{assertion*}
    Если существует хотя бы одна точка $x_0 \in I,\ W(x_0) \ne 0$, то система функций \break$y_1(x), \ldots,\ y_n(x)$ линейно независима.    
\end{assertion*}
Пусть дано ЛОДУ n-го порядка
\begin{align*}
    y^{(n)} + p_1(x) y^{(n-1)} + \ldots + p_{n-1}(x)y' + p_n(x)y = 0 \tag{1}
\end{align*}

\begin{theorem*}[О существовании ФСР ЛОДУ n-го порядка]
    Любое ЛОДУ n-го порядка (1) с непрерывными на промежутке $I$ коэффициентами $p_1(x),\, \ldots,\, p_n(x)$ имеет ФСР, то есть систему из $n$ линейно независимых функций.
\end{theorem*}
\begin{proof}
    Рассмотрим ЛОДУ n-го порядка
    \[
        y^{(n)} + p_1(x)y^{(n-1)} + \ldots + p_{n-1}(x) y' + p_n(x) y = 0
    \]
    $p_1(x),\, \ldots,\, p_n(x)$ --- непрерывны на $I$. \\
    Рассмотрим произвольный числовой определитель, отличный от нуля:
    \[
        \begin{vmatrix}
            \gamma_{11} & \gamma_{12} & \cdots & \gamma_{1n} \\
            \gamma_{21} & \gamma_{22} & \cdots & \gamma_{2n} \\
            \cdots & \cdots & \cdots & \cdots \\
            \gamma_{n1} & \gamma_{n2} & \cdots & \gamma_{nn}
        \end{vmatrix} \ne 0\qquad \gamma_{ij} \in \R,\ (ij) = \overline{1,n}
    \]
    Возьмём $\forall x_0 \in I$ и сформулируем для ЛОДУ n-го порядка задачи Коши, причём начальное условие в точке $x_0$ для $i$-ой ЗК возьмём из $i$-го столбца определителя.\\[1ex]
    \fbox{1 ЗК}: \vspace{-\topsep}
    \begin{gather*}
        y^{(n)} + p_1(x)y^{(n-1)} + \ldots + p_{n-1}(x) y' + p_n(x) y = 0\text{ --- ДУ}\\
        \left\{ \begin{aligned}
            y(x_0) &= \gamma_{11} \\
            y'(x_0) &= \gamma_{21}\\
            \cdots\cdots&\cdots\cdots \\
            y^{(n-1)}(x_0) &= \gamma_{n1} 
        \end{aligned}\right.\text{ --- начальное условие}
    \end{gather*}
    По теореме \textit{о существовании и единственности решения} 1-ая задача Коши имеет единственное решение $y_1(x)$. \\
    \[
        \cdots\cdots\cdots\cdots\cdots\cdots\cdots\cdots\cdots\cdots\cdots\cdots\cdots\cdots\cdots\cdots\cdots
    \]
    \fbox{n ЗК}:
    \begin{gather*}
        y^{(n)} + p_1(x)y^{(n-1)} + \ldots + p_{n-1}(x) y' + p_n(x) y = 0\text{ --- ДУ}\\
        \left\{ \begin{aligned}
            y(x_0) &= \gamma_{1n} \\
            y'(x_0) &= \gamma_{2n}\\
            \cdots\cdots&\cdots\cdots \\
            y^{(n-1)}(x_0) &= \gamma_{nn} 
        \end{aligned}\right.\text{ --- начальное условие}
    \end{gather*}
    По теореме \textit{о существовании и единственности решения} n-ая задача Коши имеет единственное решение $y_n(x)$. \\
    Рассмотрим функции:
    \begin{align*}
        &y_1 \text{ --- решение 1-ой ЗК} \\
        &y_2 \text{ --- решение 2-ой ЗК} \\
        &\cdots\cdots\cdots\cdots\cdots\cdots\cdots\cdot \\
        &y_n \text{ --- решение n-ой ЗК} 
    \end{align*}
    Определитель Вронского функций $y_1,\ \ldots,\ y_n\colon$
    \begin{gather*}
        \begin{vmatrix}
            \gamma_{1}(x_0) & \gamma_{2}(x_0) & \cdots & \gamma_{n}(x_0) \\
            \gamma_{2}'(x_0) & \gamma_{2}'(x_0) & \cdots & \gamma_{n}'(x_0) \\
            \cdots & \cdots & \cdots & \cdots \\
            \gamma_{1}^{(n-1)}(x_0) & \gamma_{2}^{(n-1)}(x_0) & \cdots & \gamma_{n}^{(n-1)}(x_0)
        \end{vmatrix} = \begin{vmatrix}
            \gamma_{11} & \gamma_{12} & \cdots & \gamma_{1n} \\
            \gamma_{21} & \gamma_{22} & \cdots & \gamma_{2n} \\
            \cdots & \cdots & \cdots & \cdots \\
            \gamma_{n1} & \gamma_{n2} & \cdots & \gamma_{nn}
        \end{vmatrix} \ne 0
    \end{gather*}
    По утверждению: $\exists\, x_0 \in I\colon W(x_0) \ne 0\ \Rightarrow\ y_1,\,\ldots,\, y_n$ --- линейно независимы $\Rightarrow$ $y_1,\,\ldots,\, y_n$ образуют ФСР.
\end{proof}

\newpage
\subsection{Сформулировать и доказать теорему о структуре общего решения линейного однородного ДУ n-го порядка}

\begin{theorem*}[О структуре общего решения ЛОДУ n-го порядка]
    Общим решением ЛОДУ n-го порядка
    \begin{align*}
        y^{(n)} + p_1(x)y^{(n-1)} + \ldots + p_{n-1}(x) y' + p_n(x) y = 0 \tag{1}
    \end{align*}
    с непрерывными коэффициентами $p_1(x),\, \ldots,\, p_n(x)$ на промежутке $I$ является линейная комбинация частных решений, входящих в ФСР.
    \begin{align*}
        y_{\text{оо}} = C_1y_1 + \ldots + C_ny_n \tag{2}
    \end{align*}
    <<оо>> --- общее решение однородного уравнения
    \[
        y_1,\, \ldots,\, y_n \text{ --- ФСР ЛОДУ (1)},\quad C_1,\, \ldots,\, C_n \text{ --- } const
    \]
\end{theorem*}
\begin{proof}
    1) Покажем, что (2) решение ЛОДУ (1), но не общее. Для этого подставим (2) в (1):
    \begin{align*}
        \left(C_1y_1 + \ldots + C_ny_n\right)^{(n)} &+ p_1(x)\left(C_1y_1 + \ldots + C_ny_n\right)^{(n-1)} + \\ 
        &+ \ldots + \\ 
        &+ p_{n-1}(x) \left(C_1y_1 + \ldots + C_ny_n\right)' + \\ 
        &+ p_{n}(x) \left(C_1y_1 + \ldots + C_ny_n\right) = 0
    \end{align*}
    Вычислим производные:
    \begin{align*}
        C_1y_1^{(n)} + \ldots + C_ny_n^{(n)} &+ p_1(x) C_1y_1^{(n-1)} + \ldots + p_1(x)C_ny_n^{(n-1)} + \\ 
        &+ \ldots + \\ 
        &+ p_{n-1}(x) C_1y_1' + \ldots + p_{n-1}(x)C_ny_n' + \\ 
        &+ p_{n}(x) C_1y_1 + \ldots + p_{n}(x)C_ny_n = 0
    \end{align*} \vspace{-0.5\topsep}
    Группируем:
    \begin{align*}
        &C_1\overbrace{\left(y_1^{(n)} + p_1(x)y_1^{(n-1)} + \ldots + p_{n-1}(x)y_1' + p_n(x) y_1\right)}^0 + \\ 
        + &\ldots + \\ 
        + &C_n\underbrace{\left(y_n^{(n)} + p_1(x)y_n^{(n-1)} + \ldots + p_{n-1}(x)y_n' + p_n(x) y_n\right)}_0 = 0
    \end{align*}
    Так как $y_1,\, \ldots,\, y_n$ --- частные решения ЛОДУ (1), то:
    \begin{align*}
        C_1\cdot 0 + \ldots + C_n \cdot 0 &= 0 \\
        0 &= 0\ \Rightarrow\ (2)\text{ --- решение } (1)
    \end{align*}
    2) Покажем, что (2) --- это общее решение (1), то есть из него можно выделить единственное частное решение, удовлетворяющее начальному условию:
    \begin{align*}
        \left\{ \begin{aligned}
            y(x_0) &= y_0 \\
            y'(x_0) &= y_{10} \\
            y''(x_0) &= y_{20} \\
            \cdots\cdots&\cdots\cdots \\
            y^{(n-1)}(x_0) &= y_{n\cdot10} \\
        \end{aligned} \right.\quad x_0 \in I \tag{3}
    \end{align*}
    Подставим (2) в (3):
    \begin{gather*}
        \left\{ \begin{aligned}
            y(x_0) &= C_1y_1(x_0) + \ldots + C_ny_n (x_0) = y_0 \\
            y'(x_0) &=  C_1y_1'(x_0) + \ldots + C_ny_n' (x_0) = y_{10} \\
            y''(x_0) &=  C_1y_1''(x_0) + \ldots + C_ny_n'' (x_0) = y_{20} \\
            \cdots\cdots&\cdots\cdots\cdots\cdots\cdots\cdots\cdots\cdots\cdots\cdots\cdots \\
            y^{(n-1)}(x_0) &=  C_1y_1^{(n-1)}(x_0) + \ldots + C_ny_n^{(n-1)} (x_0) = y_{n\cdot 10}
        \end{aligned}\right. \text{ --- СЛАУ}
    \end{gather*}
    СЛАУ относительно $C_1,\, \ldots,\, C_n$. Определитель этой системы --- это определитель Вронского.\vspace{-\topsep}
    \begin{gather*}
        W (x_0) = \begin{vmatrix}
            y_1(x_0) & y_2(x_0) & \cdots & y_n(x_0) \\
            y_1'(x_0) & y_2'(x_0) & \cdots & y_n'(x_0) \\
            \cdots & \cdots & \cdots & \cdots \\
            y_1^{(n-1)}(x_0) & y_2^{(n-1)}(x) & \cdots & y_n^{(n-1)}(x_0) 
        \end{vmatrix}
    \end{gather*}
    так как $y_1,\, \ldots,\, y_n$ ФСР $\Rightarrow\ y_1,\, \ldots,\, y_n$ линейно независимы $\Rightarrow\ W(x_0) \ne 0\ \Rightarrow$ ранг расширенной матрицы СЛАУ совпадает с рангом основной матрицы $\Rightarrow$ число неизвестных совпадает с числом уравнений $\Rightarrow$ СЛАУ имеет единственное решение:
    \[
        C_1^{0},\, \ldots,\, C_n^{0}
    \]
    В силу теоремы \textit{о существовании и единственности решения задачи Коши}:
    \[
        y = C_1^0y_1 + \ldots + C_n^0y_n\text{ --- единственное решение ЗК (1), (3)}
    \]
    То есть получилось из (2) выделить частное решение, удовлетворяющее начальному условию (3) $\Rightarrow$ по определению общего решения (2) --- общее решение ЛОДУ (2).
\end{proof}

\newpage
\subsection{Сформулировать и доказать теорему о структуре общего решения линейного неоднородного ДУ n-го порядка}

\begin{theorem*}[О структуре общего решения ЛНДУ]
    Общее решение ЛНДУ (1) с непрерывными на промежутке $I$ функциями\break $p_1(x),\, \ldots,\, p_n(x),\, f(x)$ равно сумме общего решения соответствующего ЛОДУ (2) и некоторого частного решения ЛНДУ (1).
    \begin{gather}
        \boxed{y_{\text{он}} = y_{\text{оо}} + y_{\text{чн}}}
    \end{gather}
    \begin{itemize}
        \item <<оо>> --- общее решение однородного уравнения
        \item <<чн>> --- частное решение неоднородного уравнения
    \end{itemize}
\end{theorem*}
\begin{proof}
    Сначала покажем, что (4) решение ЛНДУ (1), но не общее.
    Подставим (4) в (1):
    \begin{align*}
        \big(y_{\text{оо}} + y_{\text{чн}}\big)^{(n)} + p_1(x) \big(y_{\text{оо}} + y_{\text{чн}}\big)^{(n-1)} + \ldots + p_{n-1}(x) \big(y_{\text{оо}} + y_{\text{чн}}\big)' + p_n(x) \big(y_{\text{оо}} + y_{\text{чн}}\big) = f
    \end{align*}
    Вычислим производные:
    \begin{align*}
        y_{\text{оо}}^{(n)} + y_{\text{чн}}^{(n)} + p_1(x)y_{\text{оо}}^{(n-1)} + p_1(x) y_{\text{чн}}^{(n-1)} + \ldots &+ p_{n-1}(x)y_{\text{оо}}' + p_{n-1}(x) y_{\text{чн}}' + \\ 
        &+ p_{n}(x)y_{\text{оо}} + p_{n}(x) y_{\text{чн}} = f
    \end{align*}
    Группируем $y_{\text{оо}},\ y_{\text{чн}}\colon$
    \begin{align*}
        &\overbrace{y_{\text{оо}}^{(n)} + p_1(x)y_{\text{оо}}^{(n-1)} + \ldots + p_{n-1}(x)y_{\text{оо}}' + p_{n}(x)y_{\text{оо}}}^0 + \\ 
        +\, &\underbrace{y_{\text{чн}}^{(n)} + p_1(x) y_{\text{чн}}^{(n-1)} + \ldots  + p_{n-1}(x) y_{\text{чн}}' + p_{n}(x) y_{\text{чн}}}_f = f
    \end{align*}
    Так как $y_{\text{оо}}$ --- общее решение ЛОДУ (2), $y_{\text{чн}}$ --- частное решение ЛНДУ (1):
    \[
    0 + f = f\ \Rightarrow\ f = f\ \Rightarrow\ (4) \text{ --- решение ЛНДУ (1)}
    \]
    По теореме \textit{о существовании и единственности решения задачи Коши} следует, что ЗК (1), (3) имеет единственное решение.\\[1ex]
    Покажем, что (4) --- общее решение. (4) перепишется в виде:
    \begin{align*}
        y_{\text{он}} = C_1 y_1 + \dots + C_n y_n + y_{\text{чн}} \tag{4.1}
    \end{align*}
    $y_1,\, \ldots,\, y_n$ --- ФСР ЛОДУ (2)\qquad $C_1,\, \ldots,\, C_n$ --- $const$\\[1ex]
    Так как функции $p_1(x),\, \ldots,\, p_n(x),\, f(x)$ непрерывны на $I$, то по теореме \textit{о существовании и единственности решения задачи Коши} $\Rightarrow\ \exists$ единственное решение задачи Коши (1), (3). \\[1ex]
    Остаётся показать, что (4.1) и его производная удовлетворяют начальному условию (3), то есть из начального условия (3) единственным образом можно выделить константы $C_1^0,\, \ldots\, C_n^0$, то есть можно выделить частное решение. \\
    Для этого (4.1) дифференцируем $(n - 1)$ раз и подставляем в (3):
    \begin{align*}
        &\left\{ \begin{aligned}
            y_\text{он} (x_0) &= C_1y_1 (x_0) + \dots + C_ny_n (x_0) + y_{\text{чн}} (x_0) = y_0 \\
            y'_\text{он} (x_0) &= C_1y_1' (x_0) + \dots + C_ny_n' (x_0) + y_{\text{чн}}' (x_0) = y_{10} \\
            \cdot\cdots\cdots&\cdots\cdots\cdots\cdots\cdots\cdots\cdots\cdots\cdots\cdots\cdots\cdots\cdots\cdots \\
            y_\text{он}^{(n-1)} (x_0) &= C_1y_1^{(n-1)} (x_0) + \dots + C_ny_n^{(n-1)} (x_0) + y_{\text{чн}}^{(n-1)} (x_0) = y_{n\cdot 10} \\
        \end{aligned} \right. \\[1ex]
        &\left\{ \begin{aligned}
            C_1y_1 (x_0) + \dots + C_ny_n (x_0) &= y_0 - y_{\text{чн}} (x_0)\\
            C_1y_1' (x_0) + \dots + C_ny_n' (x_0) &= y_{10} - y_{\text{чн}}' (x_0)  \\
            \cdot\cdots\cdots\cdots\cdots\cdots\cdots\cdots\cdots&\cdots\cdots\cdots\cdots\cdots \\
            C_1y_1^{(n-1)} (x_0) + \dots + C_ny_n^{(n-1)} (x_0) &= y_{n\cdot 10} - y_{\text{чн}}^{(n-1)} (x_0) \\
        \end{aligned} \right. \tag{5}
    \end{align*} 
    (5) --- это СЛАУ относительно $C_1,\, \ldots,\, C_n$.\\[1ex]
    Определитель СЛАУ (5) --- это определитель Вронского:
    \begin{gather*}
        W (x_0) = \begin{vmatrix}
            y_1(x_0) & y_2(x_0) & \cdots & y_n(x_0) \\
            y_1'(x_0) & y_2'(x_0) & \cdots & y_n'(x_0) \\
            \cdots & \cdots & \cdots & \cdots \\
            y_1^{(n-1)}(x_0) & y_2^{(n-1)}(x_0) & \cdots & y_n^{(n-1)}(x_0)\\
        \end{vmatrix} \ne 0\text{ т.к. } y_1,\, \ldots,\, y_n \text{ ФСР ЛОДУ (2)}
    \end{gather*}
    Так как $W(x_0) \ne 0$, то ранг расширенной матрицы равен рангу основной матрицы (число уравнений совпадает с числом неизвестных) $\Rightarrow$ СЛАУ (5) имеет единственное решение $C_1^0,\, \ldots,\, C_n^0$. \\[1ex]
    Тогда функция $y = C_1^0 y_1 + \ldots + C_n^0y_n + y_{\text{чн}}$ --- является частным решением ЗК (1), (3), удовлетворяющим начальному условию (3) $\Rightarrow$ из решения (4.1) выделим частное решение $y = C_1^0 y_1 + \ldots + C_n^0y_n + y_{\text{чн}}$ $\Rightarrow$ по определению общего решения (\textbf{опр.\ref{опр: общее решение ДУ n-го порядка}}) (4.1) или (4) --- общее решение ЛНДУ (1).
\end{proof}

\subsection{Сформулировать и доказать теорему о наложении (суперпозиции) частных решений линейного неоднородного ОДУ}

\begin{theorem*}[О наложении (суперпозиции) частных решений ЛНДУ]
    Если \vspace{-\topsep}
    \begin{align*}
        y_1 &\text{ --- решение ЛНДУ (1) с правой частью } f_1, \\ 
        \cdot\cdot&\cdots\cdots\cdots\cdots\cdots\cdots\cdots\cdots\cdots\cdots\cdots\cdots\cdots\cdots \\
        y_n &\text{ --- решение ЛНДУ (1) с правой частью } f_n
    \end{align*}
    то линейная комбинация 
    \[
        y = C_1y_1 + \ldots + C_ny_n
    \]
    является решением ЛНДУ (1) с правой частью 
    \[
        f = C_1f_1 + \ldots + C_nf_n
    \]
\end{theorem*}
\newpage
\begin{proof}
    Так как \vspace{-\topsep}
    \begin{align*}
        y_1 &\text{ --- решение ЛНДУ (1) с правой частью } f_1, \\ 
        \cdot\cdot&\cdots\cdots\cdots\cdots\cdots\cdots\cdots\cdots\cdots\cdots\cdots\cdots\cdots\cdots \\
        y_n &\text{ --- решение ЛНДУ (1) с правой частью } f_n
    \end{align*}
    то верны равенства
    \begin{align*}
        \left\{\begin{aligned}
            y_1^{(n)} + p_1(x)y_1^{(n-1)} + \ldots + p_{n-1}(x)y_1' + p_n(x)y_1 &= f_1 \\ 
            \cdots\cdots\cdots\cdots\cdots\cdots\cdots\cdots\cdots\cdots\cdots\cdots\cdots\cdots\cdots&\cdots\cdot \\
            y_n^{(n)} + p_1(x)y_n^{(n-1)} + \ldots + p_{n-1}(x)y_n' + p_n(x)y_n &= f_n
        \end{aligned}\right. \tag{$*$}
    \end{align*}
    Рассмотрим \vspace{-\topsep}
    \begin{align*}
        y = C_1 y_1 + \ldots + C_ny_n \tag{$\vee$}
    \end{align*}
    Подставим ($\vee$) в левую часть (1):
    \begin{align*}
        \big(C_1 y_1 + \ldots + C_ny_n\big)^{(n)} &+ p_1(x)\big(C_1 y_1 + \ldots + C_ny_n\big)^{(n-1)} + \ldots + \\ 
        &+ p_{n-1} (x) \big(C_1 y_1 + \ldots + C_ny_n\big)' + p_n(x) \big(C_1 y_1 + \ldots + C_ny_n\big)
    \end{align*}
    Вычислим производные:
    \begin{align*}
        C_1 y_1^{(n)} + \ldots + C_ny_n^{(n)} &+ p_1(x)C_1 y_1^{(n-1)} + \ldots + p_1(x)C_ny_n^{(n-1)} + \ldots + \\ 
        &+ p_{n-1} (x) C_1 y_1' + \ldots + p_{n-1}(x)C_ny_n' + p_n(x) C_1 y_1 + \ldots + p_n(x)C_ny_n
    \end{align*}
    Группируем $y_1/y_n$:
    \begin{align*}
        &C_1 \overbrace{\big(y_1^{(n)} + p_1(x)y_1^{(n-1)} + \ldots + p_{n-1} (x) y_1' + p_n(x) y_1\big)}^{f_1} + \ldots + \\ 
        +\, &C_n \underbrace{\big(y_n^{(n)} + p_1(x)y_n^{(n-1)} + \ldots + p_{n-1}(x) y_n'+ p_n(x) y_n\big)}_{f_n} \xlongequal{\text{($*$)}} C_1f_1 + \ldots + C_nf_n
    \end{align*}
\end{proof}

\newpage
\subsection{Сформулировать и доказать свойства частных решений линейного однородного ДУ}

\begin{theorem*}
    Множество частных решений ЛОДУ n-го порядка (2) с непрерывными функциями $p_1(x),\ \ldots,\ p_n(x)$ на промежутке $I$ образует линейное пространство.
\end{theorem*}
\begin{proof}
    Пусть $y_1$ и $y_2$ --- частные решения ЛОДУ n-го порядка (2). Тогда:
    \begin{align*}
        +\ \begin{aligned}
            y_1^{(n)} + p_1(x)y_1^{(n-1)} + \ldots + p_{n-1}(x)y_1' + p_n(x)y_1 &= 0 \\ 
            y_2^{(n)} + p_1(x)y_2^{(n-1)} + \ldots + p_{n-1}(x)y_2' + p_n(x)y_2 &= 0
        \end{aligned}
    \end{align*}
    Складываем уравнения:
    \[
        \left(y_1^{(n)} + y_2^{(n)}\right) + p_1(x)\left(y_1^{(n-1)} + y_2^{(n-1)}\right) + \ldots + p_{n-1} (x) \left(y_1' + y_2'\right) + p_n(x)\left(y_1 + y_2\right)= 0
    \]
    По свойству производной:
    \[
        (y_1 + y_2)^{(n)} + p_1(x) (y_1 + y_2)^{(n-1)} + \ldots + p_{n-1}(x)(y_1 + y_2)' + p_n(x)(y_1 + y_2) = 0
    \]
    Обозначим $y = y_1 + y_2$:
    \[
        y^{(n)} + p_1(x)y^{(n-1)}  + p_2(x)y^{(n-2)} + \ldots + p_{n-1}(x)y' + p_n(x)y = 0
    \]
    $y = y_1 + y_2$ --- частное решение ЛОДУ (2). \\
    Пусть $y_1$ --- частное решение ЛОДУ n-го порядка (2)
    Тогда:
    \begin{gather*}
        y_1^{(n)} + p_1(x)y_1^{(n-1)}  + p_2(x)y_1^{(n-2)} + \ldots + p_{n-1}(x)y_1' + p_n(x)y_1 = 0\quad \Big| \cdot C = const,\ C \ne 0 \\
        \begin{aligned}
            C\cdot y_1^{(n)} + C\cdot p_1(x)y_1^{(n-1)} + \ldots + C\cdot p_{n-1}(x) y_1' + C\cdot p_n(x) y_1 &= 0 \\
            (C y_1)^{(n)} + p_1(x)(Cy_1)^{(n-1)} + \ldots + p_{n-1}(x)(C y_1)' + p_n(x)(C y_1) &= 0
        \end{aligned}
    \end{gather*}
    Обозначим $y = Cy_1,\quad C \text{ --- } const,\ C \ne 0$:
    \begin{gather*}
        y^{(n)} + p_1(x)y^{(n-1)} + \ldots + p_{n-1}(x)y' + p_n(x)y = 0 \\
        \Downarrow \\
        y = C\cdot y_1, \text{ где } C \text{ --- } const,\ C \ne 0\quad \text{ --- решение ЛОДУ (2)} 
    \end{gather*}
    По определению линейного пространства $\Rightarrow$ частные решения ЛОДУ n-го порядка образуют линейное пространство.
\end{proof}

\newpage
\begin{theorem*}
    Если $y_1,\ \ldots,\ y_n$ --- частные решения ЛОДУ (2), то их линейная комбинация, то есть $y = C_1y_1 + \ldots + C_ny_n$, где $C_1,\ \ldots,\ C_n$ --- $const$, является решением ЛОДУ (2).
\end{theorem*}
\begin{proof}
    Пусть $y_1,\ \ldots,\ y_n$ --- частные решения ЛОДУ (2).
    \begin{align*}
        + \left|\quad\begin{aligned}
            y_1^{(n)} + p_1(x)y_1^{(n-1)} + \ldots + p_{n-1}(x)y_1' + p_n(x)y_1 &= 0\quad \Big| \cdot C_1 \\ 
            y_2^{(n)} + p_1(x)y_2^{(n-1)} + \ldots + p_{n-1}(x)y_2' + p_n(x)y_2 &= 0\quad \Big| \cdot C_2 \\
            \cdots\cdots\cdots\cdots\cdots\cdots\cdots\cdots\cdots\cdots\cdots\cdots\cdots\cdots\cdots&\cdots\cdot \\
            y_n^{(n)} + p_1(x)y_n^{(n-1)} + \ldots + p_{n-1}(x)y_n' + p_n(x)y_n &= 0\quad \Big| \cdot C_n
        \end{aligned}\right.
    \end{align*}
    Умножим каждое уравнение на константу $C_1,\ C_2,\ \ldots,\ C_n$, где $C_i \ne 0,\ i = \overline{1,n}$.
    \begin{align*}
        \left(C_1y_1^{(n)} + C_2y_2^{(n)} + \ldots + C_ny_n^{(n)}\right) &+ p_1(x)\left(C_1y_1^{(n-1)} + C_2y_2^{(n-1)} + \ldots + C_ny_n^{(n-1)}\right) + \\ 
        &+ \ldots + \\ 
        &+ p_{n-1}(x) \left(C_1y_1' + C_2y_2' + \ldots + C_ny_n'\right) + \\ 
        &+ p_{n}(x) \left(C_1y_1 + C_2y_2 + \ldots + C_ny_n\right) = 0
    \end{align*}
    По свойству производной:
    \begin{align*}
        \left(C_1y_1 + C_2y_2 + \ldots + C_ny_n\right)^{(n)} &+ p_1(x)\left(C_1y_1 + C_2y_2 + \ldots + C_ny_n\right)^{(n-1)} + \\ 
        &+ \ldots + \\ 
        &+ p_{n-1}(x) \left(C_1y_1 + C_2y_2 + \ldots + C_ny_n\right)' + \\ 
        &+ p_{n}(x) \left(C_1y_1 + C_2y_2 + \ldots + C_ny_n\right) = 0
    \end{align*}
    Обозначим $y' = C_1y_1 + C_2y_2 + \ldots + C_ny_n$:
    \begin{gather*}
        y^{(n)} + p_1(x) y^{(n-1)} + \ldots + p_{n-1} (x)y' + p_n(x) y = 0 \\
        \Downarrow \\
        y = C_1y_1 + C_2y_2 + \ldots + C_ny_n \text{ --- решение ЛОДУ n-го порядка}
    \end{gather*}
\end{proof}

\newpage
\subsection{Вывести формулу Остроградского - Лиувилля для линейного ОДУ 2-го порядка}

Рассмотрим ЛОДУ 2-го порядка
\begin{gather}
    y'' + p_1(x)y' + p_2(x)y = 0
\end{gather}
Пусть $y_1$ и $y_2$ --- два частных решения ЛОДУ (1). Для $y_1$ и $y_2$ верны равенства:
\begin{gather*}
    \left\{ \begin{aligned}
        y_1'' + p_1(x)y_1' + p_2(x)y_1 = 0 \\
        y_2'' + p_1(x)y_2' + p_2(x)y_2 = 0 
    \end{aligned}\quad \right| \begin{aligned}
        &\cdot (-y_2) \\
        &\cdot y_1
    \end{aligned}\ \text{<<+>>}
\end{gather*}
\begin{align}
    y_1y_2'' - y_2y_1'' + p_1(x) \left(y_1y_2' - y_2 y_1'\right) + p_2(x) (\cancel{y_1y_2 - y_1y_2})= 0
\end{align}
Введём обозначение:
\[
    W(x) = \begin{vmatrix}
        y_1 & y_2 \\
        y_1' & y_2'
    \end{vmatrix} = y_1 y_2' - y_1' y_2
\]
\begin{align*}
    \Big(W(x)\Big)' = \Big(y_1y_2' - y_2 y_1'\Big)' = \cancel{y_1'y_2'} + y_1y_2'' - y_1''y_2 - \cancel{y_1'y_2'} = y_1y_2'' - y_1''y_2
\end{align*}
(2) примет вид:
\begin{align*}
    W' + p_1(x)\cdot W &= 0 \text{ ДУ с разделяющимися переменными} \\
    W' &= -p_1(x) \cdot W \\
    \dv{W}{x} &= -p_1(x) \cdot W\quad \Big|\ : W \ne 0\quad \Big| \cdot dx \\
    \frac{dW}{W} &= -p_1(x)\dd{x} \\
    \ln |W| &= - \int p_1(x)\dd{x} + C,\ \forall C \text{ --- } const \\
    e^{\ln |W|} &= e^{-\int p_1(x)\dd{x}} \cdot e^C \\
    |W| &= e^{-\int p_1(x)\dd{x}} \cdot C_1,\ \forall C_1 = e^C > 0 \\
    W &= C_2 \cdot e^{-\int p_1(x)\dd{x}},\ \forall C_2 = \pm C_1 \ne 0
\end{align*}
$W = 0$ --- особое решение
\[
    \stackunder{\boxed{W = C_3 \cdot e^{-\int p_1(x)\dd{x}},\ \forall C_3 \text{ --- } const}}{\text{Формула Остроградского-Лиувилля}}
\]
\begin{remark}
    Формула Остроградского-Лиувилля для ЛОДУ n-го порядка имеет тот же вид, что и для ЛОДУ 2-го порядка, где $p_1(x)$ --- коэффициент при $(n-1)$-ой производной при условии, что коэффициент при n-ой производной равен 1.
\end{remark}

\newpage
\subsection{Вывести формулу для общего решения линейного однородного ДУ 2-го порядка с постоянными коэффициентами в случае простых действительных корней характеристического уравнения}

\begin{align*}
    &\boxed{y'' + a_1 y' + a_2 y = 0}\quad a_1, a_2 \text{ --- } const\qquad \text{ЛОДУ} \tag{1} \\
    &\boxed{k^2 + a_1k + a_2 = 0}\qquad \text{характеристическое уравнение}\\
    &D = a_1^2 - 4a_2
\end{align*}
\fbox{1 случай} $D > 0\ \Rightarrow$ характеристическое уравнение имеет два действительных различных корня.
\[
    k_1 \ne k_2 \in \R
\]
ФСР ЛОДУ (1): \vspace{-\topsep}
\begin{gather*}
    \left\{ \begin{aligned}
        y_1 = e^{k_1x} \\
        y_2 = e^{k_2x} \\
    \end{aligned} \right.
\end{gather*}
Покажем, что $y_1$ и $y_2$ линейно независимы, то есть образуют ФСР ЛОДУ (1):
\begin{gather*}
    W(X) = \begin{vmatrix}
        y_1 & y_2 \\
        y_1' & y_2' 
    \end{vmatrix} = \begin{vmatrix}
        e^{k_1x} & e^{k_2x} \\
        k_1e^{k_1x} & k_2e^{k_2x} 
    \end{vmatrix} = e^{(k_1 + k_2) x} (k_2 - k_1) \\
    \begin{aligned}
        \left. \begin{aligned}
            e^{(k_1 + k_2) x} &\ne 0,\ \forall x \in I \\
            k_2 - k_1 &\ne 0, \text{ т.к. } k_2 \ne k_1
        \end{aligned} \right\} &\Rightarrow W(x) \ne 0\ \Rightarrow\ \text{функции } y_1 = e^{k_1x},\ y_2  = e^{k_2x} \text{ лин. зависимы } \Rightarrow \\ 
        &\Rightarrow \text{ ФСР ЛОДУ (1)}
    \end{aligned}
\end{gather*}
По теореме \textit{о структуре общего решения ЛОДУ}:
\[
    y_{\text{оо}} = C_1y_1 + C_2y_2 = C_1 e^{k_1x} + C_2 e^{k_2x}
\]

\subsection{Вывести формулу для общего решения линейного однородного ОДУ 2-го порядка с постоянными коэффициентами в случае комплексных корней характеристического уравнения}

\begin{align*}
    &\boxed{y'' + a_1 y' + a_2 y = 0}\quad a_1, a_2 \text{ --- } const\qquad \text{ЛОДУ} \tag{1} \\
    &\boxed{k^2 + a_1k + a_2 = 0}\qquad \text{характеристическое уравнение}\\
    &D = a_1^2 - 4a_2
\end{align*}
$D< 0\ \Rightarrow$ характеристическое уравнение имеет комплексные корни.
\[
    k_{1,2} = \alpha \pm \beta \cdot i\qquad i \text{ --- мнимая единица},\ \sqrt{-1} = i
\]
$\alpha$ --- действительная часть\qquad $\beta$ --- мнимая часть \\
Формула Эйлера:
\begin{gather*}
    \left\{ \begin{aligned}
        e^{i\varphi} &= \cos \varphi + i \sin \varphi \\
        e^{-i \varphi} &= \cos \varphi - i \sin \varphi
    \end{aligned} \right.
\end{gather*}
По корням характеристического уравнения находим частные решения ЛОДУ (1). \\
$k_1 = \alpha + \beta i\colon$
\[
    y_1 = e^{k_1x} = e^{(\alpha + \beta i)x} = e^{\alpha x}\cdot e^{\beta i x} = \boxed{e^{\alpha x} \left(\cos \beta x + i \sin \beta x\right)}
\]
$k_2 = \alpha - \beta i\colon$
\[
    y_2 = e^{k_2x} = e^{(\alpha - \beta i)x} = e^{\alpha x} \cdot e^{-\beta i x} = \boxed{e^{\alpha x} \left(\cos \beta x - i \sin \beta x\right)}
\]
Найдём действительные решения ЛОДУ (1). Составим линейные комбинации:
\begin{align*}
    \widetilde{y_1} = \frac{y_1 + y_2}{2} = e^{\alpha x} \cos \beta x \\
    \widetilde{y_2} = \frac{y_1 - y_2}{2i} = e^{\alpha x} \sin \beta x
\end{align*}
Из свойств частных решений ЛОДУ следует (\textbf{с.\pageref{sec: свойства частных решений ЛОДУ n-го порядка}}), что $\widetilde{y_1}$ и $\widetilde{y_2}$ --- тоже решения ЛОДУ (как линейная комбинация). \\
Покажем, что $\widetilde{y_1}$ и $\widetilde{y_2}$ линейно независимы:
\begin{align*}
    W(x) &= \begin{vmatrix}
        \widetilde{y_1} & \widetilde{y_2} \\
        \widetilde{y_1}' & \widetilde{y_2}'
    \end{vmatrix} = \begin{vmatrix}
        e^{\alpha x} \cos \beta x & e^{\alpha x} \sin \beta x \\
        \alpha e^{\alpha x} \cos \beta x - \beta e^{\alpha x} \sin \beta x & \alpha e^{\alpha x} \sin \beta x + \beta e^{\alpha x} \cos \beta x
    \end{vmatrix} = \\ 
    &= \cancel{\alpha e^{2 \alpha x} \cos \beta x \sin \beta x} + \beta e^{2 \alpha x} \cos^2 \beta x - \cancel{\alpha e^{2 \alpha x} \cos \beta x \sin \beta x} + \beta e^{2 \alpha x} \sin^2 \beta x = \\ 
    &= e^{2 \alpha x} \beta \cdot 1 \ne 0\qquad \text{т.к } e^{2 \alpha x} \ne 0\ \forall x \in I
\end{align*}
$\beta \ne 0$, так как если $\beta = 0$, то $k_1 = k_2 = \alpha$ --- действительные корни
\begin{gather*}
    \Rightarrow\ \left. \begin{aligned}
        \widetilde{y_1} &= e^{\alpha x} \cos \beta x \\
        \widetilde{y_2} &= e^{\alpha x} \sin \beta x
    \end{aligned} \right\}\ \text{линейно независимы } \Rightarrow \text{ ФСР}
\end{gather*}
По теореме \textit{о структуре решений ЛОДУ} (\textbf{Т.\ref{т: структура общего решения ЛОДУ n-го порядка}}):
\[
    y_{\text{оо}} = C_1 y_1 + C_2 y  _2 = e^{\alpha x} \left(C_1 \cos \beta x + C_2 \sin \beta x\right)
\]

\subsection{Вывести формулу для общего решения линейного однородного ОДУ 2-го порядка с постоянными коэффициентами в случае кратных корней характеристического уравнения}

\begin{align*}
    &\boxed{y'' + a_1 y' + a_2 y = 0}\quad a_1, a_2 \text{ --- } const\qquad \text{ЛОДУ} \tag{1} \\
    &\boxed{k^2 + a_1k + a_2 = 0}\qquad \text{характеристическое уравнение}\\
    &D = a_1^2 - 4a_2
\end{align*}
$D = 0\ \Rightarrow$ характеристическое уравнение имеет два действительных равных между собой корня / один корень кратности два.
\begin{gather*}
    \begin{aligned}
        k_1 = k_2 = k \in \R\\
        y_1 = e^{kx}
    \end{aligned}\quad k = -\dfrac{a_1}{2}
\end{gather*}
Найдём $y_2$ --- частное решение ЛОДУ (1) по известному частному решению $y_1$, причём $y_1$ и $y_2$ линейно независимы:
\begin{align*}
    y_2 &= y_1 \int \frac{1}{y^2_1}\cdot e^{-\int p_1(x) \dd{x}}\dd{x} = \left| \begin{aligned}
    p_1(x) = a_1 \text{ --- } const \\
    y_1 = e^{kx},\ k\in \R
    \end{aligned} \right| = e^{kx} \int \frac{1}{e^{2kx}} \cdot e^{-\int a_1\dd{x}} = \\ 
    &= e^{-\tfrac{a_1}{2}x} \int \frac{1}{\cancel{e^{-a_1x}}}\cdot \cancel{e^{-a_1x}}\dd{x} = e^{-\tfrac{a_1}{2}x} \int dx = e^{-\tfrac{a_1}{2}x}x
\end{align*}
\begin{gather*}
    \left.\begin{aligned}
        y_1 &= e^{-\tfrac{a_1}{2}x} \\
        y_2 &= xe^{-\tfrac{a_1}{2}x}
    \end{aligned}\right\}\ \text{два частных решения ЛОДУ (1)}
\end{gather*}
Покажем, что $y_1$ и $y_2$ линейно независимы:
\begin{align*}
    W(x) &= \begin{vmatrix}
        y_1 & y_2 \\
        y_1' & y_2'
    \end{vmatrix} = \begin{vmatrix}
        e^{-\tfrac{a_1}{2}x} & xe^{-\tfrac{a_1}{2}x} \\
        -\frac{a_1}{2} e^{-\tfrac{a_1}{2}x} & e^{-\tfrac{a_1}{2}x} - \frac{a_1}{2} x e^{-\tfrac{a_1}{2}x}
    \end{vmatrix} = e^{-a_1x} - \cancel{\frac{a_1}{2}x e^{-a_1x}} + \cancel{\frac{a_1}{2}x e^{-a_1x}} = \\ 
    &= e^{-a_1x} \ne 0,\ \forall x \in I\ \Rightarrow\ y_1, y_2 \text{ лин. нез. } \Rightarrow \text{ образуют ФСР } 
\end{align*}
ФСР ЛОДУ (1):
\begin{gather*}
    \left\{ \begin{aligned}
        y_1 &= e^{-\tfrac{a_1}{2}x} \\
        y_2 &= xe^{-\tfrac{a_1}{2}x} \\
    \end{aligned} \right.
\end{gather*}
По теореме \textit{о структуре общего решения ЛОДУ} (\textbf{Т.\ref{т: структура общего решения ЛОДУ n-го порядка}}):
\[
    y_{\text{оо}} = C_1y_1 + C_2y_2 = C_1 e^{-\tfrac{a_1}{2}x} + C_2 xe^{-\tfrac{a_1}{2}x}
\]

\subsection{Описать метод Лагранжа вариации произвольных постоянных для линейного неоднородного ОДУ 2-го порядка и вывести систему соотношений для варьируемых переменных}

Рассмотрим ЛНДУ 2-го порядка
\begin{align}
    y'' + p_1(x)y' + p_2(x)y &= f(x) \\
    y'' + p_1(x)y' + p_2(x)y &= 0\qquad \text{ЛОДУ}
\end{align}
$p_1(x),\, \ldots,\, p_n(x)$ --- функции.\\
Пусть $y_1$ и $y_2$ --- это ФСР ЛОДУ (2). Тогда по теореме \textit{о структуре общего решения ЛОДУ} (\textbf{Т.\ref{т: структура общего решения ЛОДУ n-го порядка}}):
\[
    y_\text{оо} = \underbrace{C_1y_1 + C_2y_2}_{\text{ФСР ЛОДУ}}\qquad C_1,\, C_2 \text{ --- } \forall const
\]
\underline{Метод Лагранжа}: предполагаемый вид решения ЛНДУ (1):
\begin{gather}
    y_{\text{он}} = C_1(x)y_1 + C_2(x)y_2
\end{gather}
$C_1(x),\, C_2(x)$ --- некоторые функции. \\
Вычислим:
\begin{align*}
    y_{\text{он}}' = C_1'y_1 + C_1y_1' + C_2'y_2 + C_2y_2' = \underbrace{C_1'y_1 + C_2'y_2}_0 + C_1y_1' + C_2y_2'
\end{align*}
\textbf{Первое дополнительное условие Лагранжа}:
\[
    \boxed{C_1'y_1 + C_2'y_2 = 0}
\]
\begin{align*}
    y_{\text{он}}' &= C_1y_1' + C_2y_2' \\
    y_{\text{он}}'' &= C_1'y_1' + C_1y_1'' + C_2'y_2' + C_2y_2''
\end{align*}
$y_{\text{он}},\, y_{\text{он}}',\, y_{\text{он}}''$ в (1):
\begin{align*}
    C_1'y_1' + C_1y_1'' + C_2'y_2' + C_2y_2'' + p_1(x) \cdot \big(C_1y_1' + C_2y_2'\big) + p_2(x) \big(C_1y_1 + C_2y_2\big) = f
\end{align*}
Группируем:
\begin{align*}
    C_1'y_1' + C_2'y_2' + C_1 \underbrace{\Big(y_1'' + p_1(x)y_1' + p_2(x)y_1\Big)}_0 + C_2 \underbrace{\Big(y_2'' + p_1(x)y_2' + p_2(x)y_2\Big)}_0 = f
\end{align*}
Так как $y_1,y_2$ --- решения ЛОДУ (2), то\\
\textbf{Второе условие Лагранжа}:
\[
    \boxed{C_1'y_1' + C_2'y_2' = f}
\]
Предполагаемое решение (3) будет являться решением ЛНДУ (1), если функции $C_1(x)$ и $C_2(x)$ удовлетворяют условиям:
\begin{align*}
    \left\{ \begin{aligned}
        C_1' y_1 + C_2'y_2 &= 0 \\
        C_1'y_1' + C_2'y_2' &= f
    \end{aligned}\right.\text{ --- \textbf{система варьируемых переменных}}
\end{align*}
Определяем из системы варьируемых переменных $C_1'(x)$ и $C_2'(x)$.
\[
    C_1'(x) = \varphi(x)\qquad C_2'(x) = \Psi (x)
\]
Интегрируем:
\begin{align*}
    C_1(x) &= \int \varphi(x)\dd{x} + k_1,\quad \forall k_1 \text{ --- } const \\
    C_2(x) &= \int \Psi(x)\dd{x} + k_2,\quad \forall k_2 \text{ --- } const
\end{align*}
Подставляем $C_1(x),\ C_2(x)$ в (3):
\begin{align*}
    y_{\text{он}} &= C_1(x) y_1 + C_2(x) y_2 = \left(\int \varphi(x)\dd{x} + k_1\right) y_1 + \left(\int \Psi (x)\dd{x} + k_2\right) y_2 = \\ 
    &= \underbrace{k_1y_1 + k_2y_2}_{y_{\text{оо}}} + \underbrace{y_1 \int \varphi(x) \dd{x} + y_2 \int \Psi(x)\dd{x}}_{y_{\text{чн}}}
\end{align*}
Система варьируемых переменных имеет единственное решение, так как определитель --- это определитель Вронского.
\begin{gather*}
    W(x) = \begin{vmatrix}
        y_1 & y_2 \\
        y_1' & y_2'
    \end{vmatrix} \ne 0 \quad\text{ т.к. } y_1 \text{ и } y_2 \text{ ФСР ЛОДУ}
\end{gather*}

\end{document}