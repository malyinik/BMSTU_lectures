\input{../../preamble2.tex}

\begin{document}
\tableofcontents
\newpage
\section{Теоретические вопросы}
\subsection{Сформулируйте определение наклонной асимптоты}
\begin{definition}

\end{definition}
\subsection{Сформулируйте определение производной функции в точке}
\begin{definition}
	\textbf{Производной функции $y = f(x)$} в точке $x_0$ называется \underline{предел} отношения приращения функции к приращению аргумента при стремлении последнего к нулю.
	\begin{align*}
		\boxed{y'(x_0) = \lim\limits_{\Delta x \to 0} \frac{\Delta y}{\Delta x}}
	\end{align*}
\end{definition}

\subsection{Сформулируйте определение односторонней производной функции}
\begin{definition}
	Производной функции $y=f(x)$ в точке $x_0$ справа(слева) или \textbf{правосторонней (левосторонней) производной} называется предел отношения приращения функции к приращению аргумента при стремлении к нулю справа(слева).
	\begin{gather*}
		\boxed{y'_+(x_0) = \lim\limits_{\Delta x \to 0+} \frac{\Delta y}{\Delta x}} \qquad \boxed{y'_-(x_0) = \lim\limits_{\Delta x \to 0-} \frac{\Delta y}{\Delta x}}
	\end{gather*}
\end{definition}

\subsection{Сформулируйте определение производной $n$-го порядка}
\begin{definition}
	\textbf{Производной $n$-го порядка} или \textbf{$n$-ой производной} функции \\ $y=f(x)$ называется производная от $(n-1)$-ой производной функции $y=f(x)$
	\begin{align*}
		\boxed{y^{(n)} = \left( y^{(n-1)} \right)'}
	\end{align*}
\end{definition}

\subsection{Сформулируйте определение дифференцируемой функции в точке}
\begin{definition}
	Функция $y=f(x)$ называется \textbf{дифференцируемой в точке $x_0$}, если существует константа $A$ такая, что приращение функции в этой точке представимо в виде: \[ \boxed{\Delta y = A\cdot \Delta x + \alpha (\Delta x) \cdot \Delta x} \]
	где $\alpha (\Delta x)$ -- б.м.ф. при $\Delta x \to 0$
\end{definition}
\newpage
\subsection{Сформулируйте определение дифференциала первого порядка}
Пусть функция $y=f(x)$ определена в окрестности точки $x_0$ и дифференцируема в точке $x_0$.\\
Тогда по определению дифференцируемой функции: \begin{align}
	\Delta y = f'(x_0) \cdot \Delta x + \alpha (\Delta x) \cdot \Delta x
\end{align}
где $\alpha (\Delta x)$ -- б.м.ф. при $\Delta x \to 0$
\begin{definition}
	\textbf{Дифференциалом} функции $y=f(x)$ в точке $x_0$ называется главная часть приращения функции $\Delta y$ или первое слагаемое в равенстве (1).
	\begin{align}
		\boxed{dy = f'(x_0) \cdot \Delta x}
	\end{align}
\end{definition}

\subsection{Сформулируйте определение дифференциала $n$-го порядка}
\begin{definition}
	$n$-ым дифференциалом или \textbf{дифференциалом $n$-го порядка} называется дифференциал от дифференциала $(n-1)$-го порядка.
	\begin{gather*}
		d^ny = d(d^{n-1}y), \quad n=2,3,\ldots
	\end{gather*}
\end{definition}

\subsection{Сформулируйте определение возрастающей функции}
\begin{definition}
	Функция $f(x)$ называется \textbf{возрастающей} на промежутке $I$, если для любых точек $x_1, x_2 \in I$, таких что $x_2 > x_1$ выполняется неравенство $f(x_2) > f(x_1)$.\\\\
	\textit{Пояснение}: Функция $f ( x )$ называется \textbf{возрастающей} на промежутке $I$, если бóльшему значению аргумента соответствует бóльшее значение функции.
\end{definition}

\subsection{Сформулируйте определение невозрастающей функции}
\begin{definition}
	Функция $f(x)$ называется \textbf{невозрастающей} на промежутке $I$, если для любых точек $x_1, x_2 \in I$ , таких что $x_2 > x_1$ выполняется неравенство $f(x_2) \le f(x_1)$. \\\\
	\textit{Пояснение}: Функция $f ( x )$ называется \textbf{невозрастающей} на промежутке $I$,
	если\\ 
    бóльшему значению аргумента соответствует не бóльшее значение функции.
\end{definition}

\subsection{Сформулируйте определение убывающей функции}
\begin{definition}
Функция $f(x)$ называется \textbf{убывающей} на промежутке $I$, если для любых точек $x_1, x_2 \in I$, таких что $x_2 > x_1$ выполняется неравенство $f(x_2) < f(x_1)$.\\\\
\textit{Пояснение}: Функция $f(x)$ называется \textbf{убывающей} на промежутке $I$, если бóльшему значению аргумента соответствует меньшее значение функции.
\end{definition}
\newpage
\subsection{Сформулируйте определение неубывающей функции}
\begin{definition}
    Функция $f(x)$ называется \textbf{неубывающей} на промежутке $I$, если для любых точек $x_1, x_2 \in I$, таких что $x_2 > x_1$ выполняется неравенство $f(x_2) \ge f(x_1)$.\\\\
    \textit{Пояснение}: Функция $f(x)$ называется \textbf{неубывающей} на промежутке $I$, если бóльшему значению аргумента соответствует не меньшее значение функции.
\end{definition}

\subsection{Сформулируйте определение монотонной функции}
\begin{definition}
    Возрастающая, убывающая, невозрастающая и неубывающая функции называются \textbf{монотонными}.
\end{definition}

\subsection{Сформулируйте определение строго монотонной функции}
\begin{definition}
    Возрастающая и убывающая функции называются \textbf{строго монотонными}.
\end{definition}

\subsection{Сформулируйте определение локального минимума}
\begin{theorem*}
    Функция $f(x)$ имеет локальный минимум в точке $x_0$, если существует окрестность $U(x_0)$ точки $x_0$ такая, что для любого $x \in U(x_0)$ выполняется неравенство: \[ f(x)\ge f(x_0) \]
\end{theorem*}
\begin{definition}
    Точка $x_0$, в которой функция $f(x)$ имеет локальный минимум, называется точкой \textbf{локального минимума} этой функции.
\end{definition}

\subsection{Сформулируйте определение строгого локального минимума}
\begin{theorem*}
    Функция $f(x)$ имеет строгий локальный минимум в точке $x_0$, если существует окрестность $U(x_0)$ точки $x_0$ такая, что для любого $x \in U(x_0)$ выполняется неравенство: \[ f(x) > f(x_0) \]
\end{theorem*}
\begin{definition}
    Точка $x_0$, в которой функция $f(x)$ имеет строгий локальный минимум, называется точкой \textbf{строгого локального минимума} этой функции.
\end{definition}
\newpage
\subsection{Сформулируйте определение локального максимума}
\begin{theorem*}
    Функция $f(x)$ имеет локальный максимум в точке $x_0$, если существует окрестность $U(x_0)$ точки $x_0$ такая, что для любого $x \in U(x_0)$ выполняется неравенство: \[ f(x) \le f(x_0) \]
\end{theorem*}
\begin{definition}
    Точка $x_0$, в которой функция $f(x)$ имеет локальный максимум, называется точкой \textbf{локального максимума} этой функции.   
\end{definition}

\subsection{Сформулируйте определение строгого локального максимума}
\begin{theorem*}
    Функция $f(x)$ имеет строгий локальный максимум в точке $x_0$, если существует окрестность $U(x_0)$ точки $x_0$ такая, что для любого $x \in U(x_0)$ выполняется неравенство: \[ f(x) < f(x_0) \]
\end{theorem*}
\begin{definition}
    Точка $x_0$, в которой функция $f(x)$ имеет строгий локальный максимум, называется точкой \textbf{строгого локального максимума} этой функции.
\end{definition}

\subsection{Сформулируйте определение экстремума}
\begin{definition}
    Минимум, максимум, строгий минимум, строгий максимум функции $f(x)$ называются \textbf{экстремумами} этой функции.
\end{definition}
\begin{definition*}
    Точка $x_0$, в которой функция $f(x)$ имеет экстремум, называется \textbf{точкой экстремума} этой функции.
\end{definition*}
\subsection{Сформулируйте определение строгого экстремума}
\begin{definition}
    Строгий минимум и строгий максимум функции $f(x)$ называются \textbf{строгими экстремумами} этой функции.
\end{definition}
\begin{definition*}
    Точка $x_0$, в которой функция $f(x)$ имеет строгий экстремум, называется \textbf{точкой строго экстремума} этой функции.
\end{definition*}

\subsection{Сформулируйте определение стационарной точки}
\begin{definition}
    Точка $x_0$, в которой производная функции $f(x)$ равна нулю, называется \textbf{стационарной точкой} этой функции.
\end{definition}
\newpage
\subsection{Сформулируйте определение критической точки}
\begin{definition}
    Точка $x_0$, в которой производная функции $f(x)$ равна нулю или не существует, называется \textbf{критической точкой} этой функции.
\end{definition}

\subsection{Сформулируйте определение выпуклости функции на промежутке}
\begin{definition}
    Пусть функция $f(x)$ определена на интервале $(a;b)$. Функция $f(x)$ называется \textbf{выпуклой вверх (вниз)} на этом интервале, если любая точка касательной, проведённой к графику функции $f(x)$ (кроме точки касания) лежит выше (ниже) точки графика функции $f(x)$ с такой же абсциссой.
\end{definition}

\subsection{Сформулируйте определение точки перегиба графика функции}
\begin{definition}
    Пусть функция $f(x)$ определена на интервале $(a;b)$.\\
    Точка $x_0 \in (a;b)$ называется \textbf{точкой перегиба \underline{функции}} $f(x)$, если эта функция непрерывна в точке $x_0$ и если существует число $\delta > 0$ такое, что направление выпуклости функции $f(x)$ на интервалах $(x_0 - \delta, x_0)$ и $(x_0, x_0 + \delta)$ различны.\\
    При этом точка $\big(x_0, f(x_0)\big)$ называется \textbf{точкой перегиба \underline{графика функции}} $f(x)$.\\\\
    \textit{Пояснение}: Точка $x_0 \in (a;b)$ является \textbf{точкой перегиба \underline{функции}} $f(x)$, если при переходе через эту точку, направление выпуклости функции меняется на противоположное.
\end{definition}
\newpage
\section{Теоретические вопросы (формулировки теорем)}

\subsection{Сформулируйте необходимое и достаточное условие наличия наклонной асимптоты}
\begin{theorem}

\end{theorem}

\subsection{Сформулируйте необходимое и достаточное условие дифференцируемости функции в точке}
\begin{theorem}[Необходимое и достаточное условие дифференцируемости функции]\hlabel{Условие дифференцируемости}
	Функция $y=f(x)$ дифференцируема в точке $x_0$ тогда и только тогда, когда она имеет в этой точке конечную производную.
\end{theorem}

\subsection{Сформулируйте теорему о связи дифференцируемости и непрерывности функции}
\begin{theorem}[Связь дифференцируемости и непрерывности функции] \hlabel{Связь диф. и непр. функции}
	Если функция дифференцируема в точке $x_0$, то она в этой точке непрерывна.
\end{theorem}

\subsection{Сформулируйте теорему о производной произведения}
\begin{theorem}[Производная произведения]
	Пусть функции $u = u(x),\ \upsilon = \upsilon (x)$ дифференцируемы в точке $x$.\\
	Тогда в этой точке дифференцируемо их произведение и справедливо равенство:
	\begin{gather*}
		(u \cdot \upsilon)' = u'\cdot \upsilon + \upsilon' \cdot u
	\end{gather*}
\end{theorem}

\subsection{Сформулируйте теорему о производной частного}
\begin{theorem}[Производная частного]
	Пусть функции $u = u(x),\ \upsilon = \upsilon (x)$ дифференцируемы в точке $x$.\\
	Тогда в этой точке дифференцируемо их частное и справедливо равенство:
	\begin{gather*}
		\left(\dfrac{u}{\upsilon}\right)' = \dfrac{u' \cdot \upsilon - \upsilon' \cdot u}{\upsilon^2}
	\end{gather*}
\end{theorem}

\subsection{Сформулируйте свойство инвариантности формы записи дифференциала первого порядка}
\begin{theorem}[Инвариантность формы первого дифференциала]
	Форма записи первого дифференциала не зависит от того, является ли $x$ независимой переменной или функцией другого аргумента.
\end{theorem}
\newpage
\subsection{Сформулируйте теорему Ферма}
\begin{theorem}[\textbf{Теорема Ферма} (о нулях производной)]\hlabel{Ферма}
	Пусть функция $y=f(x)$ определена на промежутке $X$ и во внутренней точке $\bm{c}$ этого промежутка достигает наибольшего или наименьшего значения. Если в этой точке существует производная $f'(c)$, то $f'(c) = 0$.
\end{theorem}

\subsection{Сформулируйте теорему Ролля}
\begin{theorem}[\textbf{Теорема Ролля}]\hlabel{Ролль}
	Пусть $y=f(x)$
	\begin{enumerate}
		\item непрерывна на $[a;b]$
		\item дифференцируема на $(a;b)$
		\item $f(a) = f(b)$
	\end{enumerate}
	Тогда $\exists\ c \in (a;b)\colon f'(c) = 0$
\end{theorem}
% \newpage
\subsection{Сформулируйте теорему Лагранжа}
\begin{theorem}[\textbf{Теорема Лагранжа}]
	Пусть функция $y=f(x)$
	\begin{enumerate}
		\item непрерывна на $[a;b]$
		\item дифференцируема на $(a;b)$
	\end{enumerate}
	Тогда $\exists\ c \in (a;b)\colon \boxed{f(b) - f(a) = f'(c) \cdot (b-a)}$
\end{theorem}

\subsection{Сформулируйте теорему Коши}
\begin{theorem}[\textbf{Теорема Коши}]
	Пусть функции $f(x)$ и $\varphi (x)$
	\begin{enumerate}
		\item непрерывны на $[a;b]$
		\item дифференцируемы на $(a;b)$
		\item $\forall x \in (a;b)\colon \varphi' (x) \ne 0$
	\end{enumerate}
	Тогда $\exists\ c \in (a;b)\colon$ \vspace{-\topsep}
	\begin{gather*}
		\boxed{\frac{f(b) - f(a)}{\varphi (b) - \varphi (a)} = \frac{f'(c)}{\varphi'(c)}}
	\end{gather*}
\end{theorem}

\end{document}

