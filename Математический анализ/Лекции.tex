\input{../../preamble2.tex}
\begin{document}

\begin{titlepage}
    \vspace*{0pt}
    \vfill
    \centering
    \Huge\textbf{Математический анализ} \\[7pt]
    \Large\textbf{Лекции} \\
    \large 1 семестр \\ 
    \vfill
    \begin{flushright}
        \normalsize GitHub: \href{https://github.com/malyinik}{malyinik} \\
    \end{flushright}
    \normalsize 2023 г.
\end{titlepage}
\newpage

\tableofcontents
\newpage

\begin{center}
\large{\textbf{Модуль №1} \\
\textit{Элементарные функции и пределы}}
\end{center}

\input{Основы математического анализа.tex}
\newpage
\zerocounter
\input{Числовая последовательность.tex}
\newpage
\zerocounter
\input{Предел функции.tex}
\newpage
\zerocounter
\input{бмф. Арифметические операции.tex}
\newpage
\zerocounter
\input{Предел функции. бмф ббф.tex}
\newpage
\zerocounter
\input{Непрерывность функции. Точки разрыва.tex}
\newpage
\zerocounter
\begin{center}
\large{\textbf{Модуль №2} \\
\textit{Дифференциальное исчисление функции одной переменной}}
\end{center}
\input{Производная функции.tex}
\newpage
\zerocounter
\input{Дифференциал функции.tex}
\zerocounter
\newpage
\input{Основные теоремы дифференциального исчисления.tex}
\newpage
\zerocounter
\input{Раскрытие неопределённостей.tex}
\zerocounter
\newpage
\input{Формула Тейлора.tex}
\zerocounter
\newpage
\input{Исследование функции.tex}

\end{document}