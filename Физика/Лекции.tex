\input{../../preamble2.tex}

\begin{document}

\begin{titlepage}
    \vspace*{0pt}
    \vfill
    \centering
    \Huge\textbf{Физика} \\[7pt]
    \Large\textbf{Лекции} \\
    \small (конспект по учебнику) \\
    \large 2 семестр\\ 
    \vfill
    \begin{flushright}
        \normalsize GitHub: \href{https://github.com/malyinik}{malyinik} \\
    \end{flushright}
    \normalsize 2024 г.
\end{titlepage}
\newpage

\tableofcontents
\newpage

%%%%%%%%%%%

% Угловые скорость и ускорение твёрдого тела. 
% Классический закон сложения скоростей и ускорений при поступательном движении системы отсчета.

\section{Кинематика}

\subsection{Механическое движение}

\begin{definition}
    Совокупность тел, выделенная для рассмотрения, называется \textbf{механической системой}
\end{definition}

\begin{definition}
    Совокупность неподвижных друг относительно друга тел, по отношению к которым рассматривается движение, и отсчитывающих время часов образует \textbf{систему отсчёта}.
\end{definition}

\begin{note}
    Движение одного и того же тела относительно различных систем отсчёта может иметь разных характер.
\end{note}

\begin{definition}
    Тело, размерами которого в условиях данной задачи можно пренебречь, можно пренебречь, называется \textbf{материальной точкой}.
\end{definition}

\begin{definition}
    \textbf{Абсолютно твёрдым телом} называется тело, деформациями которого в условиях данной задачи можно пренебречь.
\end{definition}

\subsubsection{Виды движения}

\begin{definition}
    \textbf{Поступательное движение} --- это такое движение, при котором любая прямая, связанная с движущимся телом, остаётся параллельной самой себе.
\end{definition}

\begin{definition}
    При \textbf{вращательном движении} все точки тела движутся по окружностям, центры которых лежат на одной и той же прямой, называемой \textbf{осью вращения}. Ось вращения может находиться вне тела.
\end{definition}

% рисунок на с. 19

\subsubsection{Некоторые сведения о векторах}

\begin{definition}
    \textbf{Радиусом-вектором} $\bv{r}$ некоторой точки называется вектор, проведённый из начала координат в данную точку. Его проекции на координатные оси равны декартовым координатам данной точки:
    \begin{gather*}
        r_x = x,\quad r_y = y,\quad r_z = z
    \end{gather*}
    Радиус-вектор представим в виде линейной комбинации ортов $\bv{e_x,\ e_y,\ y_z}$:
    \begin{align*}
        \bv{r} &= x\bv{e_x} + y\bv{e_y} + z\bv{e_z} \\
        r^2 &= x^2 + y^2 + z^2
    \end{align*}
\end{definition}

\newpage
\subsubsection{Производная вектора}

\hspace{\parindent}Рассмотрим вектор, который изменяется со временем по закону $\bv{a} (t)$. Проекции этого вектора на координатные оси представляют собой заданные функции. Следовательно:
\begin{align}
    \bv{a} (t) = \bv{e_x} a_x (t) + \bv{e_y} a_y (t) + \bv{e_z} a_z (t)
\end{align}

Пусть за промежуток времени $\Delta t$ проекции вектора получают приращения $\Delta a_x,\ \Delta a_y,\ \Delta a_z$. Тогда вектор получит приращение $\Delta \bv{a} = \bv{e_x}\Delta a_x,\ \bv{e_y}\Delta a_y,\ \bv{e_z}\Delta a_z$. Скорость изменения вектора $\bv{a}$ со времени можно охарактеризовать отношением $\Delta \bv{a}$ к $\Delta t$:
\begin{align}
    \frac{\Delta \bv{a}}{\Delta t} = \bv{e_x}\frac{\Delta a_x}{\Delta t} + \bv{e_y}\frac{\Delta a_y}{\Delta t} + \bv{e_z}\frac{\Delta a_z}{\Delta t}
\end{align}
Это отношение даёт среднюю скорость изменения $\bv{a}$ в течение промежутка времени $\Delta t$.
\begin{definition}
    \textbf{Скорость изменения вектора} $\bv{a}$ в момент времени $t$ равна пределу отношения (2), получающемуся при неограниченном уменьшении $\Delta t$:
    \begin{gather*}
        \bv{a} = \lim_{\Delta t \to 0} \frac{\Delta \bv{a}}{\Delta t} = \bv{e_x} \lim_{\Delta t \to 0} \frac{\Delta a_x}{\Delta t} + \bv{e_y} \lim_{\Delta t \to 0} \frac{\Delta a_y}{\Delta t} + \bv{e_z} \lim_{\Delta t \to 0} \frac{\Delta a_z}{\Delta t}
    \end{gather*}
\end{definition}
Если есть некоторая функция $f(t)$ аргумента $t$, то предел отношения приращения функции $\Delta f$ к приращению аргумента $\Delta t$, получающийся при стремлении $\Delta t$  к нулю, называется производной функции $f$ по $t$ и обозначается символом $df/dt$.
\begin{gather}
    \frac{d \bv{a}}{d t} = \bv{e_x} \frac{d a_x}{d t} + \bv{e_y} \frac{d a_y}{d t} + \bv{e_z} \frac{d a_z}{d t}
\end{gather}

В физике принято производные по времени обозначать символом соответствующей величины с точкой над ним, например,
\begin{gather*}
    \frac{d \varphi}{dt} = \dot{\varphi},\quad \frac{d^2 \varphi}{dt^2} = \ddot{\varphi},\quad \frac{d \bv{a}}{dt} = \dot{\bv{a}},\quad \frac{d^2 \bv{a}}{dt^2} = \ddot{\bv{a}}
\end{gather*}
Воспользовавшись таким обозначением формуле (3) можно придать вид
\begin{gather}
    \dot{\bv{a}} = \bv{e_x} \dot{a}_x + \bv{e_y} \dot{a}_y + \bv{e_z} \dot{a}_z
\end{gather}
Если в качестве $\bv{a}(t)$ взять радиус-вектор $\bv{r} (t)$ движущейся точки, то:
\begin{gather}
    \dot{\bv{r}} = \bv{e_x} \dot{x} + \bv{e_y} \dot{y} + \bv{e_z} \dot{z},
\end{gather}
где $x$, $y$, $z$ суть функции от $t$: $x = x(t),\ y = y(t),\ z = z(t)$.

\zerocounter
\subsection{Скорость}

\begin{definition}
    Материальная точка при своём движении описывает некоторую линию. Эта линия называется \textbf{траекторией}. 
\end{definition}
В зависимости от формы траектории различают прямолинейное движение, движение по окружности, криволинейное движение и т. п.
\newpage
\noindent Пусть материальная точка переместилась вдоль некоторой траектории из точки $1$ в точку $2$.\\
\begin{definition}
    Расстояние между точками $1$ и $2$, отсчитанное вдоль траектории, называется \textbf{путём} $s$, пройденным материальной точкой.
\end{definition}
\begin{definition}
    Прямолинейный отрезок, проведённый из точки $1$ в точку $2$, называется \textbf{перемещением} материальной точки. Перемещение $\bv{r}$ --- это вектор.
\end{definition}

\begin{definition}
    Если за равные, сколь угодно малые промежутки времени частица проходит одинаковые пути, движение частицы называют \textbf{равномерным}.
\end{definition}

Разобьём траекторию на бесконечно малые участки длины $ds$. Каждому из участков сопоставим бесконечно малое перемещение $d\bv{r}$. Разделив это перемещение на соответствующий промежуток времени $dt$, получим \textbf{мгновенную скорость} в данной точке траектории.
\begin{gather}
    \boxed{\bv{v} = \lim_{\Delta t \to 0} \frac{\Delta \bv{r}}{\Delta t} = \frac{d\bv{r}}{dt} = \dot{\bv{r}}}
\end{gather}
\textbf{Вывод:} \textit{скорость} есть производная радиуса вектора материальной точки по времени.
\begin{gather}
    \begin{aligned}
        \upsilon = |\bv{v}| = \left| \lim_{\Delta t \to 0} \frac{\Delta \bv{r}}{\Delta t} \right| = \lim_{\Delta t \to 0} \frac{|\Delta \bv{r}|}{\Delta t} & \\
        \lim_{\Delta t \to 0} \frac{\Delta s}{|\Delta \bv{r}|} = 1 &
    \end{aligned}\ \Rightarrow\ \boxed{\upsilon = \lim_{\Delta t \to} \frac{\Delta s}{\Delta t} = \frac{ds}{dt}}
\end{gather}
\textbf{Вывод:} \textit{модуль скорости} равен производной пути по времени. 
\begin{gather*}
    \begin{aligned}
        \bv{v} &= \upsilon_x \bv{e}_x + \upsilon_y \bv{e}_y + \upsilon_z \bv{e}_z \\
        \dot{\bv{r}} &= \dot{x} \bv{e}_x + \dot{y} \bv{e}_y + \dot{z} \bv{e}_z
    \end{aligned}\ \Rightarrow\ \boxed{
        \begin{aligned}
            \upsilon_x = \dot{x},\quad \upsilon_y = \dot{y},\quad \upsilon_z = \dot{z} \\ 
            \upsilon = \sqrt{\dot{x}^2 + \dot{y}^2 + \dot{z}^2}\qquad
        \end{aligned}}
\end{gather*}
\textbf{Вывод:} проекция вектора скорости на координатную ось равна производной по времени соответствующей координаты движущейся материальной точки. \\[1ex]
Вектор скорости можно представить в виде $\bv{v} = \upsilon\bv{e_\upsilon}$. Введём орт касательной к траектории $\bm{\tau}$, направив его в ту же сторону, что и $\bv{v}$. Орты $\bv{e_\upsilon}$ и $\bm{\tau}$ совпадут, поэтому:
\begin{gather}
    \bv{v} = \upsilon \bv{e_\upsilon} = \upsilon \bm{\tau}
\end{gather}
\textbf{Путь}, проходимый материальной точкой за промежуток времени от $t_1$ до $t_2$ равен
\begin{gather}
    \boxed{s = \lim\limits_{\Delta t_i \to 0} \sum_{i=1}^{N} \upsilon_i \Delta t_i = \int_{t_1}^{t_2} \upsilon (t)\, dt}
\end{gather}
Если взять интеграл не от модуля скорости, а от самой скорости $\bv{v}(t)$, то получится \textbf{вектор перемещения} материальной точки из точки, в которой она была в момент $t_1$, в точку, в которой она оказалась в момент $t_2$:
\begin{gather}
    \boxed{\int_{t_1}^{t_2}\bv{v}(t)\, dt = \int_{t_1}^{t_2} d\bv{r} = \bv{r_{12}}}
\end{gather}
\textbf{Среднее значение модуля скорости} за время от $t_1$ до $t_2$ по определению равно
\begin{gather}
    \boxed{\langle\upsilon\rangle  = \frac{s}{t_2 - t_1} = \int_{t_1}^{t_2} \upsilon (t)\, dt}
\end{gather}
Аналогично вычисляются средние значения любых скалярных или векторных функций. Например, \textbf{среднее значение скорости} равно
\begin{gather}
    \boxed{\langle \bv{v} \rangle = \frac{1}{t_2 - t_1} \int_{t_1}^{t_2} \bv{v} (t)\, dt = \frac{\bv{r_{12}}}{t_2 - t_1}}
\end{gather}

\zerocounter
\subsection{Ускорение}

\begin{definition}
    Быстрота изменения вектора $\bv{v}$, как и быстрота изменения любой функции времени, определяется производной вектора $\bv{v}$ по $t$.
    \begin{gather}
        \boxed{\bv{a} = \lim_{\Delta t \to 0} \frac{\Delta \bv{v}}{\Delta t} = \frac{d\bv{v}}{dt} = \dot{\bv{v}} = \ddot{\bv{r}}}
    \end{gather}
    Эта величина называется \textbf{ускорением}.
\end{definition}
Ускорение $\bv{a}$ играет по отношению к $\bv{v}$ такую же роль, какую вектор $\bv{v}$ играет по отношению к радиусу-вектору $\bv{r}$.
\begin{gather*}
    \begin{aligned}
        a_x &= \frac{d\upsilon_x}{dt} = \dot{\upsilon}_x \\
        \upsilon_x &= \dot{x} = \frac{dx}{dt}
    \end{aligned}\ \Rightarrow\ a_x = \frac{d\upsilon_x}{dt} = \frac{d}{dt} \left(\frac{dx}{dt}\right) = \frac{d^2x}{dt^2} = \ddot{x}\ \Rightarrow\ \boxed{a_x = \ddot{x},\quad a_y = \ddot{y},\quad a_z = \ddot{z}}
\end{gather*}
Подставим в формулу (1) выражение для $\bv{v}$:
\begin{gather}
    \boxed{\bv{a} = \dot{\upsilon} \bm{\tau} + \upsilon \dot{\bm{\tau}}}
\end{gather}
Следовательно, вектор $\bv{a}$ можно представить в виде суммы двух составляющих.

\begin{definition}
    Первая из составляющих ускорения коллинеарна с $\bm{\tau}$, то есть направлена по касательной к траектории, и поэтому обозначается $\bv{a_\tau}$ и называется \textbf{тангенциальным ускорением}.
    \begin{gather}
        \boxed{\bv{a_\tau} = \dot{\upsilon}\bm{\tau}}
    \end{gather}
\end{definition}

\begin{definition}
    Вторая из составляющих ускорения направлена по нормали к траектории и поэтому обозначается $\bv{a_n}$ и называется \textbf{нормальным ускорением}.
    \begin{gather}
        \boxed{\bv{a_n} = \upsilon \dot{\bm{\tau}} = \frac{\upsilon^2}{R}\bv{n}}
    \end{gather}
    где $\bv{n}$ --- орт нормали к траектории, направленный в ту сторону, в которую поворачивается вектор $\bm{\tau}$ при движении материальной точки по траектории.
\end{definition}
\newpage
\begin{definition}
    Степень искривлённости плоской кривой характеризуется \textbf{кривизной} $\bm{C}$, которая определяется выражением:
    \begin{gather}
        C = \lim_{\Delta s \to 0} \frac{\Delta \varphi}{\Delta s} = \frac{d\varphi}{ds}
    \end{gather}
    где $\Delta \varphi$ --- угол между касательными к кривой в точках, отстоящих друг от друга на $\Delta s$. Кривизна определяется скорость поворота касательной при перемещении вдоль кривой.
\end{definition}

\begin{definition}
    Величина, обратная кривизне $C$, называется \textbf{радиусом кривизны} $\bm{R}$.
    \begin{gather}
        R = \frac{1}{C} = \lim_{\Delta \varphi \to 0} \frac{\Delta s}{\Delta \varphi} = \frac{ds}{d\varphi}
    \end{gather}
\end{definition}

\begin{definition}
    Радиус кривизны представляет собой радиус окружности, которая сливается в данном месте с кривой на бесконечно малом её участке. Центр такой окружности называется \textbf{центром кривизны} для данной точки кривой.
\end{definition}

%%%%%%%%%%%

\end{document}